\documentclass[12pt]{report}
\usepackage{verbatim}

\title{Toward a Foundation of Mathematics More Suitable for Education}
\author{Ciarán Dunne}

\begin{document}

\begin{minipage}[h]{0.9\textwidth}
\maketitle
\begin{abstract}
Set theory as a foundation of mathematics has been studied intensely in the past century. Zermelo-Frankel with Choice (ZFC) set theory is widely accepted as a mathematical foundation, and is built parsimoniously to formalize a single notion of a set, such that every mathematical object can be constructed as a set. However, when learning introductory university courses in theoretical computer science and set theory, then notion of set, natural numbers, and ordered pairs are usually introduced as primitive concepts. We seek to give this intuition a formalisation, by constructing a model according to this specification, which can interpreted in ZFC and also characterized by a list of axioms. We look at the use of \emph{urelements} to model objects which we consider to be distinct from sets, and we also consider ways of dealing with undefinedness that can result from erroneous expressions. This paper will outline the construction of this model, its advantages and disadvantages in a mathematical and set theoretical context, and also in an educational context.
\end{abstract}
\end{minipage}
\clearpage

\chapter{Introduction}

% Abstract, Aims, Objectives, Project Description
% Are these clearly expressed, testable, and achievable?

\section{Aims}
\section{Objectives}
\section{Project Description}


\chapter{Literature Review}

% - How relevant is the literature that is covered?
% • Is there missing material?
% • Is it well structured?
% • Are good quality sources used and properly cited?
% • How strong are the comparative and critical aspects?
% • Is the literature review of an appropriate length?


\chapter{Research Questions}

% • Are the requirements/hypothesis/research questions clearly expressed, testable, and achievable?
% • Are the requirements of an appropriate length?

\chapter{Testing and Evalutation Strategy}

% • Is a suitable evaluation planned?
% • Is a sound/rigorous methodology being proposed?

\chapter{Project Plan}

% • Is a realistic project plan and timetable proposed?
% • Has a risk analysis been performed and sensible mitigation plans proposed?
% • Is there a safe core to the project, with scope for more challenging activities?
% • Does the student show a good understanding of the PLES issues relevant to the project and discussed these?

\end{document}
