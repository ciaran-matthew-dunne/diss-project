%!TEX spellcheck
%!TEX root=diss.tex
\documentclass[11pt]{report}
\usepackage{verbatim}
\usepackage[utf8]{inputenc}
\usepackage{amsmath, amsthm, amssymb, amsfonts}
\usepackage{setspace}
\usepackage{wrapfig}
\usepackage{graphicx}
\usepackage{enumitem}
\usepackage{quoting}
\usepackage{array}
\usepackage{url,hyperref}
\usepackage[margin=1in]{geometry}
\newcommand{\all}[1]{\forall_{\mathit{#1}}\hspace{0.5mm}}
\newcommand{\ex}[1]{\exists_{\mathit{#1}}\hspace{0.5mm}}
\newcommand{\eqdef}{\equiv_\mathit{def}}
\newcommand{\pleft}{\mathrel{\pi_1}}
\newcommand{\pright}{\mathrel{\pi_2}}
\newcommand{\pair}[2]{\langle #1,#2 \rangle}
\newcommand{\zin}{\mathrel{\widehat{\in}}}
\newcommand{\zpright}{\mathrel{\widehat{\pi}_2}}
\newcommand{\zpleft}{\mathrel{\widehat{\pi}_1}}
\theoremstyle{definition}
\theoremstyle{theorem}
\theoremstyle{lemma}
\newtheorem{definition}{Definition}[section]
\newtheorem{theorem}{Theorem}[section]
\newtheorem{lemma}[theorem]{Lemma}
\SetEnumitemKey{ncases}{itemindent=!,before=\let\makelabel\ncasesmakelabel}
\newcommand*\ncasesmakelabel[1]{Case #1}


\author{Ciarán Dunne}

\doublespacing
\begin{document}

\begin{titlepage}
   \begin{center}
       \vspace*{1cm}

       \Large
       \textbf{Toward a Foundation of Mathematics More Suitable for Education}

       \vspace{0.5cm}
       \large
        Final Year Dissertation

       \vspace{1.5cm}

       \textbf{Ciarán Dunne}\\
       Supervised by: J. B. Wells
       \vfill

       Submitted for the Honours Degree of \\
       Bachelor of Science in Mathematics and Computer Science

       \vspace{0.8cm}

       School of Mathematics and Computer Science\\
       Heriot-Watt University\\
       Edinburgh, United Kingdom\\
   \end{center}
\end{titlepage}

\begin{minipage}[c]{0.9\textwidth}
\begin{abstract}
\noindent
Set theory has been studied intensely in the past century, as mathematicians have searched to give mathematics a rigorous foundation.  
The study of set theory itself has led to many interesting results concerning the nature and philosophy of mathematics, as well as fundamental questions of computer science.
Zermelo-Fraenkel set theory ($\mathit{ZF}$) was developed in the early $20^{\mathit{th}}$ century to create a theory of sets free of paradoxes.
$\mathit{ZF}$ and other set theories are built as axiomatic systems with a single fundamental relation of set membership.
A consequence of this high level of abstraction is that the only objects of the theory are sets, requiring us to study mathematical objects as sets governed by the axioms of $\mathit{ZF}$.
Whilst this generality may be convenient for studies into the nature of mathematics, there still remains a large difference between the use of set theory by ordinary mathematicians and set theorists.   
In most undergraduate mathematical texts, sets are introduced as collections of non-set objects.
Simple examples in discrete mathematics often use sets of natural numbers, or even simply sets of coloured balls, to help guide intuition of the operations of set-theoretic membership, union, and intersection.
Ordered pairs are introduced as a structure distinct from sets, often used in mathematics to define co-ordinate systems, as well as for rigorous definitions of functions and relations.
This project aims to detail the modifications necessary to the theory of $\mathit{ZF}$ to allow ordered pairs as a primitive structure, as well as proving that these modifications are logically sound. 
We hope that this will begin to bridge the gap between a widely accepted foundation of mathematics, and standard mathematical practice. 
These modifications are guided by research into axiomatic set theory, in particular, theories that involve \emph{urelements} to model objects which are considered to be distinct from sets.
\end{abstract}
\end{minipage}
\clearpage

\begin{minipage}[c]{0.9\textwidth}
  \centering{\textbf{Declaration}} \\
  I, Ciarán Dunne, confirm that this work submitted for assessment is my own and is expressed in
  my own words. Any uses made within it of the works of other authors in any form (e.g.,
  ideas, equations, figures, text, tables, programs) are properly acknowledged at any point of
  their use. A list of the references employed is included.\footnotemark
  \vspace{1cm}
  Signed: \includegraphics[scale=0.25]{sig}  \\
  \vspace{1cm}
  Date: 22/11/2018
\end{minipage}
\footnotetext{This declaration is taken from Appendix C.1 in the Dissertation Handbook~\cite{dissguide}.}
\clearpage

\tableofcontents

\chapter{Introduction and Project Description}
% Abstract, Aims, Objectives, Project Description
% Are these clearly expressed, testable, and achievable?
\section{Motivation}
Since the 1800s, mathematicians and philosophers have been interested in finding logical foundations for mathematics. Various systems have been proposed, but set theory appears to have played the biggest role in this search for rigour.
Set theory is a branch of mathematics that studies sets, which can informally be described as collections of objects that possess some common property, for instance, the set of all natural numbers, denoted by $\mathbb{N}$.
Set theory proves itself to be incredibly useful in almost every area of mathematics, because it acts as a powerful language to express concepts in these fields.
The original formulation of the theory --- now known as na\"ive set theory --- given by Georg Cantor in the late 19th century, was defined somewhat formally, but used natural language to describe sets and their operations.
However, due to paradoxes discovered in na\"ive set theory, most notably Russell's paradox (``the set of all sets that do not contain themselves"), mathematicians not only questioned the correctness of their work, but also the limitations of formalism.
A proposed solution to this crisis of uncertainty was \emph{Hilbert's program}, whose main goal was to provide a complete, and consistent foundation for all of mathematics.
However, in 1931, the publication of \emph{G\"odel's incompleteness theorems} showed this goal to be impossible, by proving that in any sufficiently powerful consistent theory, there are statements which are true, but cannot be proved by the theory itself.
Though the original goals Hilbert's program were not met, many areas of mathematical logic can be viewed as natural continuations of the program, namely the formalisation of mathematics in set theory.
After decades of research, set theory has become a standard informal language for mathematicians.
Set theory also finds many applications in computer science. 
For example, the $\mathit{Mizar}$ proof assistant and the $\mathit{Z}$ specification language are both based on formal set theory \cite{mizar} \cite{zspec}.\\

Set theory and various areas of mathematics --- arithmetic, group theory, geometry --- can be defined as \emph{axiomatic systems}, with the now most widely accepted being Zermelo-Fraenkel set theory ($\mathit{ZF}$).
An axiomatic system is an explicitly stated set of axioms from which theorems can be derived~\cite{wolfaxiom}.
An axiom is a statement which is presumed to be true.
An axiomatic system is thus a collection of elementary truths, and using logical and mathematical principles, we can draw conclusions and gain new truths, known as theorems.
Axiomatic systems were one of the main subjects of Hilbert's program mentioned above, most notably the work of Turing, Church, and G\"odel, relating to the \emph{Entscheidungsproblem} posed by David Hilbert~\cite{stancomput}.\\


The axioms of $\mathit{ZF}$ are stated in the language of \emph{first-order logic}.
First-order logic uses object variables, predicate and function symbols, quantifiers and logical connectives to form logical statements.
The axioms of $\mathit{ZF}$ characterise the notion of set in terms of the \emph{membership relation}, commonly denoted by the symbol $\in$.
These axioms allow us to formally prove the existence and properties of various sets.
For instance, the first axiom of $\mathit{ZF}$ is the axiom of extensionality, which asserts that two sets are equal if and only if they contain the same elements.
The axiom schema of specification tells us that given some existing set $S$, and some logical property, there exists another set containing exactly the elements of $S$ that satisfy that property.
This ability to `filter' a set is a powerful and understandable notion, and has been mimicked in programming languages such as Python and Haskell. 
More of the axioms will be detailed in later sections.\\

In most theories, the only objects considered are the empty set, denoted $\emptyset$, and all other sets that can be formed from it, e.g $\{\emptyset, \{\emptyset, \{\emptyset\}\}\}$. 
This means that \emph{everything} is a set, so the words `object' and `set' are synonymous.
When viewed on a low-level, these sets are just collections of other sets, but with the correct definitions, one can define high-level mathematical objects such as the natural numbers, which can be defined as $0 = \emptyset$, $1=\{\emptyset\}$, $2=\{\emptyset, \{\emptyset\}\}$, and so on.
These sets are defined in this way to implement the structure and true statements of the natural numbers, for instance, they are ordered by the membership relation, since $0\in 1$, $2\notin 1$, etc.
One can then go on to define addition and multiplication, giving a rigorous basis and foundation for arithmetic.

Ordered pairs are used widely in mathematics due to the characteristic property of having a specific ordering --- i.e $(x,y) \neq (y,x)$.
This makes them an ideal candidate for co-ordinate systems and definitions of functions and relations.
However, these objects are not a primitive structure in any standard set theory, and are defined as sets.
The first such definition was given by Wiener as $\{\{\{x\},\emptyset\},\{\{y\}\}\}$, and later improved by Kuratowski as $\{\{x\},\{x,y\}\}$.
The introduction of ordered pairs is extremely important, since from it we can define relations and functions, and describe properties such as \emph{transitivity} for relations, or \emph{injectivity} for functions.
Concepts such as relations and functions play an extremely important role in formalising mathematics.\\

In education, set theory is often introduced to students at an undergraduate level, mainly to mathematics students as a tool to --- as mentioned above --- define concepts in areas such as analysis and algebra.
Set theory is often also introduced to undergraduate computer science students, as an insight into the foundations of the subject.
In these cases, set theory is rarely taught in an axiomatic way, but instead, a more practical approach is often employed. 
Sets are considered to be collections of pre-existing, non-set mathematical objects that students are familiar with, and ordered pairs are structures that can contain mathematical objects, or the newly introduced sets.  
This method allows teachers to gloss over difficult details, and focus on the applications of set theory in the field in question.
Various alternative set theories have been created such as KPU (Kripke-Platek with Urelements), and ZFA (Zermelo-Fraenkel with Atoms), to allow such non-set objects explicitly.
These objects are known as \emph{urelements} (from the German prefix \emph{ur}, meaning primordial), and are sometimes called atoms.
The non-set objects in these theories, however, have no internal structure, so ordered pairs still must be defined as sets.
For these reasons, it is necessary to create a set theory which has urelements with internal structure that behaves in the same way of that of ordered pairs. 
This begins to provide a formal basis for how we actually teach set theory in the classroom. \\

As well as an educational motivation, this project also contributes to the effort of creating a library of mathematical knowledge, understandable to both humans and machines.
In 2016, Stephen Wolfram --- creator of the influential system Mathematica --- organised a workshop to pool the knowledge of a select group of experts, in an attempt to find an agreement on how to design a language to describe all of mathematics. 
The resulting white paper states that the workshop not only failed to come to an agreement, but also failed to provide a specific direction for a first step. 
They agreed to abandon focus on the possible development of such a language, and instead, discuss the general problem of working towards a Global Digital Mathematical Library. 
The white paper then identifies criteria that such a library would have, one of which is that ``the relation between the human and machine-readable forms of the mathematics should be clear and explicit".
The importance of this is that we want to keep any potential computerization of mathematics as close as possible to how it was written. 
There are many papers in which ordered pairs are considered to be primitive, under the assumption that the foundations can be adjusted to allow this. 
Computerizing one of these papers in $\mathit{ZF}$ would mean that the original mathematics would have to be changed, losing historical accuracy, and potentially mathematical accuracy. 
The white paper also states a criteria of ``making explicit all structures in mathematics''. 
A strong example of this sort of reasoning can be found in work on the $\lambda$-calculus by Kamareddine~\cite{fairouz}, which makes the notion of substitution explicit in the theory, rather than defined at the meta-level.
This modification allows for an easier implementation of the $\lambda$-calculus in programming languages.
However, the substitution rules then become dissimilar to those used informally.\\

This project is an investigation into how we can formalise the way that set theory is taught in undergraduate mathematics and computer science.
This mainly involves a formal modification of $\mathit{ZF}$ set theory which allows ordered pairs as a primitive structure, as well as a proof of its consistency.
In addition to this, this document will act as a guide for those who want to understand the necessary modifications required for adding new objects to set theories. 
There will also be discussion of the wider philosophical relevance, and future work that could be done to further narrow the gap between theoretical foundations, and high-level mathematics. 

\section{Turtles All The Way Down}
When discussing foundations of mathematics, we are not suggesting that any particular `foundation' --- whether it be set theory, type theory --- is \emph{actually} the stuff that mathematics is made out of, we are only saying that mathematics \emph{can} be expressed in a particular formal system. 
Whilst a reductionist approach may seem tempting and even more secure, it also brings the inconvenience of having to break every single calculation down to its fundamentals.
If we had some particular foundation, in which we claimed was \emph{the} foundation for mathematics, natural questions would be: how do we know that this foundation is correct? Do we need a foundation for this foundation? 
Mathematics would still go on, since definitions and theorems can still be made, without any formal language and all-encompassing theory behind them, as long as there is some agreement or convention between those who use it.
Those working on number theory do not need to think of numbers as sets, and app developers don't consider their programs to be proofs, as the \emph{Curry-Howard Correspondence} would have you believe. 
To summarise, I find the word `foundation' is ambiguous and misleading, as it loosely implies that there exists some final, bottom layer to mathematics, which clearly leads to infinite regress. 
In the same way that a field of mathematics seeks to describe some real-world system, so-called `foundations` attempt to describe mathematics itself.\\

% \begin{wrapfigure}{r}{7cm}
% \centering
% \includegraphics[width=7cm]{turtles.jpg}
% \end{wrapfigure}
\noindent
There is a famous anecdote, in which a scientist is giving a public lecture on the Earth's place in the solar system.
The scientist is interrupted by an old lady at the back of the hall who refutes the scientist's theory to give his own: 
``We live on a crust of earth which is on the back of a giant turtle."
The scientist laughs and asks, ``If your theory is correct, then what does this turtle stand on?''
The old lady replies, ``That's a very good question. The first turtle stands on the back of a second, far larger turtle.''
``But what does this second turtle stand on?'' the scientist persists.
To this the old lady finally says: 
``It's no use dear --- it's turtles all the way down".

This anecdote is a metaphor for the problem of infinite regress, and is often used as an argument for the non-existence of an ultimate foundation of knowledge. 
If the world of mathematics rests upon a turtle of set theory, and set theory rests upon a turtle of first-order logic, then who does she rest upon? 
Set theory can also be expressed using type theory, or category theory. And also first-order logic can be expressed in set theory.
What does our turtle structure look like? 
Is it \emph{turtles all the way down}? 
An circle of turtles with an infinite radius?
I prefer the viewpoint of \emph{any turtle you like}. 

\section{Aims}
We aim to formalise an alternative set theory, to a specification that is useful in an educational context, whilst maintaining the same amount of theoretical power that $\mathit{ZF}$ has.
This theory will involve the introduction of ordered pairs as a primitive structure, distinct from sets.
These new ordered pairs should be careful in the same ways as the commonly used Kuratowski definition.
The sets and ordered pairs of this theory should have no type restrictions, in that anything can be contained in a set, and anything can be contained in an ordered pair.
This modification will require research into axiomatic set theory, in order to sufficiently modify Zermelo-Fraenkel set theory to allow ordered pairs that satisfy list of desired properties.

To be acceptable for use in an educational context, the theory must at least have sufficient power to express the high-level mathematical concepts taught in undergraduate mathematics and computer science courses.\\

In addition to these broad aims, we also state some technical aims. 
We want the theory to be a \emph{conservative extension} of $\mathit{ZF}$, in that it extends the theory with new symbols and new axioms, but proves nothing new about pure sets.
The motivation for this aim is that the new theory should be an acceptable replacement of $\mathit{ZF}$ for those who are familiar with $\mathit{ZF}$, but want to rigorously reason about ordered pairs without any worries of violating the abstraction.
An example of a conservative extension of a set theory is Von Neumann-Bernays-G\"odel set theory, which adds ``classes'' (collections of sets that are `too big' to be sets) to $\mathit{ZFC}$ ($\mathit{ZF}$ with the axiom of choice). 

Once we have specified our theory as a list of axioms, we will attempt to find an model of our theory in $\mathit{ZF}$, which will be a collection of sets in $\mathit{ZF}$ that satisfy the axioms of our theory.
If $\mathit{ZF}$ is assumed to be consistent, and an interpretation can exist, then we can conclude that our new system is also consistent.

The supervisor of this project, Joe Wells, teaches Heriot-Watt University's ``Foundations 2'' course, which uses set theory as a foundation for the theory of computability.
The course lecture notes are intentionally ambiguous on the definition of ordered pairs. 
However, discussions with students often assume ordered pairs are primitive objects that are distinct from sets.
This is not the only abuse of foundation made, there are many other shortcuts made in the course to ease the understanding for many students without a mathematical background.  
He often feels uncomfortable about teaching a course on mathematical foundations, without any confirmed foundation involving primitive ordered pairs! 
Having a guarantee that there exists a consistent mathematical system of the specification detailed above would help bring confidence into the introduction of these concepts.

From all of the mathematical work, we want to create a document that details the construction of an alternative set theory with non-set objects, so that mathematicians can easily make their own, and be able to understand the properties of it.

An additional aim and overall outcome of this project is to increase my own personal knowledge of mathematical logic, and the foundations of mathematics.
Whilst this is a personal ambition of mine, the knowledge and skills gained from this dissertation will prepare me to undertake my PhD project starting in July which will focus on a framework for the computerisation of mathematics.

It was also originally an aim of this project to introduce a formal description mechanism, to allow denotation of objects of the form ``the unique object $x$ such that $\varphi$'' in first-order formula.
In order to deal with the cases where this description mechanism failed (when no object exists, or that object is not unique), we originally planned to introduce a so-called ``garbage" or ``exception" object which would be returned in case of error.
This object, whilst belonging to the domain of any model of the theory, would not be able to be contained in sets or pairs.
Whilst this would have been a great feature and a further tool for the use of set theory in education, this development will be left to future work.

\section{Objectives}\label{objectives}
To achieve the aims specified above, we targeted 4 main objectives:
\begin{enumerate}
  \item \textbf{Provide a literature review} and documentation detailing a background of axiomatic set theory, model theory, and set theory with urelements, as well as detailing important contributions that will assist this project.
  Success of this objective has been evaluated through discussion with my supervisor, who is an expert in mathematical logic.
  In addition to this, Fairouz Kamareddine, another expert in mathematical logic, will proof-read the review.  
  The key success criteria is outlined by ensuring that the review is:
  \begin{enumerate}
    \item \textbf{Mathematically Accurate:} The review has no mathematical inconsistencies, and has as little ambiguity as possible, without going into too much detail. 
    \item \textbf{Well-Cited:} Definitions and results mentioned in the review have citations where appropriate, allowing the reader to conduct further reading if needed. 
    \item \textbf{Understandable:} This project is aimed at students with an interest in the foundations of mathematics. Concepts are introduced in a fashion similar to that of university lecture notes, without any assumptions of considerable prerequisite knowledge. 
  \end{enumerate} 

  \item \textbf{Design and present an axiomatisation} of a variant of $\mathit{ZF}$ theory with primitive ordered pairs. 
  Success of this objective has been evaluated by giving proofs that the axioms satisfy the requirements given in~\ref{zfpreq}.
  In addition to this, discussions with my supervisor have ensured that the axiomatisation is readable and intuitive. 
  The key success criteria is outlined by ensuring that the axiomatisation is: 
  \begin{enumerate}
    \item \textbf{Understandable:} The formulation and presentations of the axioms have been ensured to make clear what logical properties the axioms state. Further discussion is also given for explanation of why specific axioms are needed, and why they are stated in the ways that they are. 
    \item \textbf{Powerful:} The axioms are able to prove all of the logical properties given in the requirements. 
    \item \textbf{Interpretable:} The language and axioms of the system have been designed in order to allow the objects and relations of the theory to be easily defined in $\mathit{ZF}$.
    \item \textbf{Consistent:} Whilst it is not possible to give a syntactic proof of consistency, the system is designed to avoid contradictions. 
    \item \textbf{$\mathbf{ZF}$-like}: The axiomatisation is as close to $\mathit{ZF}$ as possible, in attempt to preserve the nice properties that $\mathit{ZF}$ has.
    Resemblance to such a popular foundation also increases understandability.  
  \end{enumerate}  

  \item \textbf{Define an intended model} as a structure with a domain in $\mathit{ZF}$, and relation symbols defined in the language of $\mathit{ZF}$. 
  This structure will have a relation symbol for the set membership relation of the theory, as well as relation symbols for the ordered pair left and right projections.
  Success of this objective will consist of a mathematical proof showing that the defined structure satisfies the axioms specified in~\ref{zfpaxioms}.
  The key success criteria for the building of the intended model is outline by ensuring it has the following properties:
  \begin{enumerate}
    \item \textbf{Satisfaction:} Every axiom of the theory given in~\ref{zfpaxioms} must be satisfied by the model. This will require a proof for each axiom, and in turn prove that the theory is consistent. 
    \item \textbf{Large enough} to contain all of the sets that are characterised by the axioms of $\mathit{ZF}$.
  \end{enumerate} 

  \item \textbf{Define an embedding} which maps all of the sets of an assumed model of $\mathit{ZF}$, to `pure sets' (those not containing any primitive ordered pairs anywhere in their structure) of the new theory. 
  This will provide an argument for the new theory being a conservative extension of $\mathit{ZF}$.
  Success of this objective will consist of a series of lemmas and a theorem showing that the mapping preserves the truth of the membership relation of $\mathit{ZF}$. 
  The key success criteria for the definition of this mapping is outlined by ensuring it is:
  \begin{enumerate}
    \item \textbf{Total:} The mapping is total and well-defined on the model of $\mathit{ZF}$, so that we know all sets of $\mathit{ZF}$ can be remade in the `pure set' model. 
    \item \textbf{Elementary:} An elementary embedding is a map which preserves the truth of a model, so that all true formula in the model of $\mathit{ZF}$ are also true in the new model.   
  \end{enumerate} 
\end{enumerate}


\chapter{Background \& Literature Review}
This chapter will review some of the main concepts that will be involved in the project, in reference to literature involving them.

\section{Set Theory}

% - How relevant is the literature that is covered?
% • Is there missing material?
% • Is it well structured?
% • Are good quality sources used and properly cited?
% • How strong are the comparative and critical aspects?
% • Is the literature review of an appropriate length

\section{Propositional and First-order Logic}
Propositional logic --- or the propositional calculus --- is an elementary system of logic which deals with statements that are true or false, and serves as a basis for logical reasoning. 
Introduction of the modern propositional calculus is usually attributed to George Boole, and Augustus DeMorgan. 
However, it was later found that unpublished work of Leibniz from the $17^{\text{th}}$ century show that he was conducting similar work.

The variables of the propositional calculus only range over two values, true and false. 
These variables are known as \emph{atomic propositions}. 
Formulas are made from atomic propositions and logical connectives combining them. 
Specific notation differs through history, but the logical connectives are commonly given as below, found in university lecture notes \cite[p.~2]{beckert} as follows:
$$\neg, \wedge, \vee,\Rightarrow,\Leftrightarrow$$
Their meanings, respectively, are `not', `and', `or', `implies', and `if and only if'. 
These connectives have rules that are well-defined and correspond to their intended meanings, for example $p \wedge q$ is true if and only if $p$ and $q$ are true.  
Propositional logic has the nice property that it is \emph{decidable}, meaning that we can determine the truth or falsity of any statement written in its language by some effective method.

First-order logic extends this by introducing \emph{predicates}, as well as universal and existential \emph{quantification}. 
The variables in first-order logic are assumed to range over some \emph{domain of discourse}.
Predicates are conditions on variables which yield true or false. Which include, for example:  
$$P(x) : x \text{ is prime}$$
$$ x > y : x \text{ is greater than } y$$
Quantifiers allow us to make statements of the form ``for all $x$ such that $\varphi$" and ``there exists $y$ such that $\psi$", formally denoted by $(\forall x:\varphi)$ and $(\exists y:\psi)$. 
Statements that use multiple instances universal or existential quantification can sometimes be shortened from $(\forall x:(\forall y: \varphi))$ to $(\forall x,y: \varphi)$ for notational convenience.
These quantifiers were first introduced by Frege \cite{frege} in 1879.
The alphabet of first-order logic consists of logical symbols and non-logical symbols.
Uniqueness can also be expressed, and is often denoted using a `new' quantifier $\exists!$, which stands for ``there exists a unique''. 
This quantifier can be defined in terms of the others by the abbreviation:
$$\exists!x:\varphi \eqdef \exists x: (\varphi \wedge (\forall y: \varphi[y/x] \Rightarrow y=x))$$
The notation $\varphi[y/x]$ represents the formula obtained from the formula $\varphi$ by proper substitution of the variable $x$ for the variable $y$.

The logical symbols are the connectives from the propositional calculus, along with the quantifiers, an equality symbol, an infinite set of variables $x_0, x_1, x_2,\ldots$, and parentheses.
The non-logical symbols are those that represent the predicates, functions, and constants that allow us to form a language for talking about the objects which the variables of the language range over.

\subsection{Signatures}
A formal description of non-logical symbols is
given by a \emph{signature} \cite[ch. 1.1]{selinger}. A signature is a triple $(F,R,\alpha)$, where:
\begin{itemize}
\item $F$ is a set of function symbols, those that represent operations that convert objects into new ones, such as $+$ or $\sqrt{\hspace{1mm}}$.
\item $R$ is a set of predicate symbols, those that represent relationships between objects, such as $>$ or $P$ as shown above.
\item $\alpha$ is a function which assigns \emph{arities} to each of the function and predicate symbols.
For example $\alpha(+) = 2$. Function symbols with arity $0$ are often called \emph{constants}, and are used to denote specific objects in the domain in question.
\end{itemize}
Signatures are usually denoted by the Greek letter $\sigma$, with subscript indicating the use of the signature in question.
Often signatures are not explicitly given as a structure, but instead simply mentioned as a `choice' of function and relation symbols.
See~\cite{shortermodel} for an alternative description. 
In the following chapters we will opt for the explicit choice of signature description, to avoid any ambiguity when working with more than one language.

A standard example is the signature for groups.
A group is an algebraic structure consisting of a set of objects and a binary operation.
A group always has an identity element in its set of objects, for instance, in the group of integers under addition, denoted $(\mathbb{Z}, +)$, $0$ is the identity element since $n + 0 = n$ for all integers $n\in\mathbb{Z}$.
Groups also always have inverses for each element, for example, we know that for each $n\in\mathbb{Z}$, there exists a unique inverse, denoted $-n$, such that $n + (-n) = 0$.

So in the signature for groups, we only have function symbols, $*$ for the generic binary operation, $^{-1}$ for the inverse operator, and $e$ as a constant for the identity element.

$$\sigma_{\mathit{grp}} = \big(\{*,^{-1}, e\},\emptyset,\{(*, 2), (^{-1}, 1),(e,0)\}\big)$$
It should be noted however, that these are just the symbols for a language, the signature does not contain any information about the actual existence of an identity element, or inverses.

The signature for ZF --- and most other set theories --- has no function symbols, and a single binary predicate symbol for set membership.
$$\sigma_{\mathit{ZF}} = \big(\emptyset,\{\in\},\{(\in, 2)\})$$

\subsection{First-Order Languages}
Using a signature $\sigma$ and the standard logical symbols of first-order logic, we can create a \emph{first-order language} \cite[ch.~1]{stanmodel}, often denoted $\mathcal{L}_\sigma$.

The \emph{terms} of a language are the variables --- for which we use Latin letters --- or constants, or the result of a function application $f(t_1,\ldots, t_n)$, where $f$ is a function symbol, and $t_1,\ldots,t_n$ are terms \cite[ch.~1.3]{selinger}.

The \emph{formula} of a language are defined \cite[ch~1.4]{selinger} by the following rules:
\begin{itemize}
  \item If $P$ is an $n$-ary predicate symbol, and $t_1,\ldots,t_n$ are terms, then $P(t_1,\ldots,t_n)$ is a formula.
  \item If $t_1$ and $t_2$ are terms, then the equality $t_1 = t_2$ is a formula.
  \item If $\varphi$ is a formula, then the negation $\neg\varphi$ is also a formula, and $(\varphi)$ is a formula.
  \item If $\varphi$ and $\psi$ are formulas, then all of the following are formulas: $\varphi\wedge\psi$, $\varphi\vee\psi$, $\varphi\Rightarrow\psi$, and $\varphi\Leftrightarrow\psi$.
  \item If $\varphi$ is a formula, and $x$ is a variable, then $\forall x:\varphi$ and $\exists x:\varphi$ are formulas.
\end{itemize}
The terms of a language represent objects in a language, whereas the formulas represent statements about objects that are either true or false. 

In the special when a function or relation symbol is binary (of arity 2), we usually write terms and formula formed from them using \emph{infix} notation.
This is commonplace in mathematics, for example, we always prefer the term $x+y$ and the formula $x>y$, rather than $+(2,2)$ and $>(x,y)$.

An occurrence of a variable contained in a formula is said to be \emph{bound} if it is quantified by either $\forall$, or $\exists$.
Otherwise the variable is said to be \emph{free}. For example, in the formula $(\forall x: x+y=0)$, $x$ is bound, and $y$ is free.

A \emph{sentence} is a formula with no free variable occurrences.
Since in sentences, all variables must be quantified, sentences can always have well-defined truth values, depending on the meaning of the function and predicate symbols.
More detail on the specification of free variables is given by Selinger \cite[ch.~1.9]{selinger}.\\

\noindent
We can use first-order languages to talk about the properties of mathematical systems.
Continuing with our earlier examples of the signatures for groups and ZF, we can use the language of groups $\mathcal{L}_{\sigma_{\text{grp}}}$ to state the property of associativity:
$$\forall x,y,z: x*(y*z) = (x*y)*z$$
and we can use the language of set theory $\mathcal{L}_{\sigma_{\text{ZF}}}$ to state the property of extensionality, which says that two sets are equal if they contain the same elements\footnote{The converse follows from the logical principle that equal objects satisfy exactly the same properties.}:
$$\forall A,B: (\forall x:(x\in A \Leftrightarrow x\in B) \Rightarrow A = B)$$

\section{Zermelo-Fraenkel Set Theory}
As discussed in the project introduction, the discovery of paradoxes in Cantor's original specification of set theory called for a re-evaluation of set theory.
Many different approaches have been taken to formalise set theory, including axiomatic, logistic, and intuistic approaches.
The most prominent and fruitful of these has been the axiomatic method.
Axioms are used as a way to formulate basic truths about the existence of various sets, and their properties.
First order logic then allows us to reason and prove properties of sets.

Before giving the axiomatic definition of $\mathit{ZF}$ set theory, we will first give a brief overview of the basic notions of set theory.
These are concepts and operations which would usually be introduced to students at an undergraduate level.
Sets are introduced as unordered collections of objects, without repetition. 
However, this is not wholly true. 
In a written representation of a set, the set being represented is invariant under the ordering and uniqueness of its elements.
It is usually necessary to stress the fact that a set is not dependent on the order its elements are written in, to avoid confusion in computer science students, who are used to using \emph{arrays} in programming languages.

Sets are commonly denoted by comma separated lists using braces, also known as ``curly brackets". 
For example, the set containing exactly the numbers $1,2$, and $3$ is denoted by $\{1,2,3\}$. 
This notation originates from Cantor~\cite{cantor}.
Infinite sets are denoted in the same way that one would informally write an infinite series, by writing some initial fragment of the series, and leaving the reader to use their intuition to guess what follows. 
For example, the set of all even numbers is informally denoted as $\{2,4,6,8,\ldots\}$.  

The fundamental notion of set theory is that of \emph{set membership}, characterised by the relation denoted by the symbol $\in$.
We write $1\in\{1,2,3\}$ to mean ``$1$ is contained in the set $\{1,2,3\}$", which is clearly a true statement. 
The negation of this relation is denoted by putting a backslash through the symbol, for example $0\notin\{1,2,3\}$ for ``$0$ is not contained in the set $\{1,2,3\}$''.
It is possible for a set to have no members.
This set is known as the \emph{empty set}, and is denoted by the symbol $\emptyset$. 

Another commonly used notion is that of \emph{subset}.
We say that a set $A$ is a subset of another set $B$ --- denoted $A \subseteq B$ --- if all of the elements of $A$ are also contained in $B$.
There are also certain operations defined on sets, known as intersection, union, and difference, denoted by $\cap$, $\cup$, and $\setminus$ respectively.
The intersection of two sets $A$ and $B$ --- $A\cap B$ --- is the set containing all of the elements which are contained in both $A$ and $B$.
The union of $A$ and $B$ --- $A\cup B$ --- is the set containing all of the elements contained in either $A$ or $B$. 
The set difference $A\setminus B$ is the set containing all of the elements of $A$ that are not in $B$.

The sets that are first introduced to students are often various sets of numbers.
At the lowest level we have the natural numbers, $\mathbb{N} = \{1,2,3,\ldots\}$, and then the integers $\mathbb{Z} = \{\ldots, -1, 0, 1, \ldots\}$.
Continuing, we have the rationals $\mathbb{Q}$, which are all numbers which can be represented as a fraction $\frac{a}{b}$, where $a$ and $b$ are integers, and $b\neq 0$.
Extending even further, we have the real numbers $\mathbb{R}$, where we find numbers such as $\sqrt{2}$, and $\pi$. 
Finally, we have the complex numbers $\mathbb{C}$, which are all numbers of the form $a+bi$, where $a$ and $b$ are real numbers, and $i$ is the imaginary number representing $\sqrt{-1}$.
These sets are usually considered to be ordered by the subset relation\footnote{Due to the way that these sets are actually defined, formally, this statement is false. It is the case however that the reals contain a subset that is the same size as, and \emph{behaves} in the same way as the natural numbers.}, that is: 
$$\mathbb{N} \subseteq \mathbb{Z} \subseteq \mathbb{Q} \subseteq \mathbb{R} \subseteq \mathbb{C}$$

These concepts are all extremely useful in mathematics, and provide a common language for discussing mathematical reasoning. 
However, as mentioned above, there is a need for a rigorous definition of set theory, to ensure that no paradoxical sets can be proven to exist. 
To continue with the notation and definitions above, we will specify the set theory $\mathit{ZF}$ as a first-order theory.
Note that this is different from the system $\mathit{ZFC}$, due to the lack of the axiom of choice --- which will be omitted for simplicity.

The signature for ZF consists of a single predicate symbol $\in$ for set membership, where `$x \in X$' is interpreted as `$x$ is contained in $X$'.
The axioms of $\mathit{ZF}$ are a set of sentences in $\mathcal{L}_{\sigma_{\mathit{ZF}}}$ that characterise the behaviour of the membership relation.
Thus, the theory of $\mathit{ZF}$ is all of the theorems that can be derived from the axioms.

\subsection{Axioms of $\mathit{ZF}$}
\textit{Foundations of Set Theory} by Fraenkel, Bar-Hillel, and Levy \cite{foundations} gives a detailed description of the axioms of the theory of $\mathit{ZF}$, as well as philosophical discussion of set theory itself.
The axioms are listed as follows:

\subsubsection*{I. Axiom of Extensionality}
If two sets have exactly the same members, then they are equal:
$$\forall x,y:(\forall z: z\in x \Leftrightarrow z\in y) \Rightarrow x=y$$

\subsubsection*{II. Axiom of Pairing}
If two sets $x$, and $y$ exist, then so does $\{x,y\}$, the set containing exactly $x$ and $y$ as members.
$$\forall x,y: \exists z: \forall u: u\in z \Leftrightarrow (u=x \vee u=y)$$
Given sets $a$ and $b$, the axiom of pairing allows us to create the set $\{a,b\}$, and also, by pairing $a$ and $a$ (or $b$ and $b$), we can create the sets $\{a\}$ and $\{b\}$.

\subsubsection*{III. Axiom of Union}
For any set $x$, there exists a set $y$, that contains all of the members of the members of $x$.
$$\forall x: \exists y: \forall z: z\in y \Leftrightarrow \exists t: t\in x \wedge z\in t$$
This set is called the \emph{union} of $x$, and is denoted $\bigcup x$.
For example, if $x = \{\{a,b\}, \{c,d,e\}\}$, then $\bigcup x = \{a,b,c,d,e\}$.

\subsubsection*{IV. Axiom Schema of Specification}
The subset relation --- denoted by $x\subseteq y$, for `$x$ is a subset of $y$' --- is defined as: $(\forall a: (a\in x \Rightarrow a\in y))$. This is when all of the members of $x$ are contained in $y$, but not necessarily the other way around.\\
\noindent
For any set $x$, and for any condition $\varphi$, there exists a set $y$ containing only the members $z \in x$ such that $\varphi$.
$$\forall x: \exists y: \forall z: z\in y \Leftrightarrow (z \in x \wedge \varphi)$$
where $x$ and $y$ are not free variables in $\varphi$.
This axiom allows us to specify subsets of a given set using some predicate. The subset of $x$ specified by the predicate $\varphi$ is denoted by $x_\varphi$ for shorthand, and by $\{a\in x \mid \varphi(a)\}$ in \emph{set comprehension} notation. 
For example, if we have the set $x = \{1,2,3,4,5,6,7\}$, and the predicate $\varphi(a) \equiv \text{`a is even'}$, by the axiom, we prove the existence of $x_\varphi = \{2,4,6\}$.
The axiom schema can also be used to prove the existence of the \emph{empty set}, denoted $\emptyset$. 
Using a universally false predicate such as $\psi(x) \equiv x\neq x$, and applying the axiom schema of specification, we obtain a set with no members, which is unique by extensionality . 

\subsubsection*{V. Axiom of Power Set}
For any set $x$, there exists a set $y$, whose members are all of the subsets of $x$.
$$\forall x: \exists y: \forall z: z\in y \Leftrightarrow z \subseteq x$$
This set is called the \emph{power set} of $x$, denoted by $\mathcal{P}(x)$.
For example, if $x = \{a,b,c\}$, then $\mathcal{P}(x) = \{\{a,b,c\},\{a,b\},\{a,c\},\{b,c\}, \{a\}, \{b\}, \{c\}, \emptyset\}$.

\subsubsection*{VI. Axiom of Infinity}
There exists a set $z$ such that, $\emptyset \in z$, and, if $x\in z$, then $x \cup \{x\}\in z$.
$$\exists z: \emptyset\in z \wedge \forall x: (x\in z \Rightarrow x \cup \{x\}\in z)$$
Axioms II-V will only allow us to prove the existence of infinitely many finite sets, but never a set with infinitely many members. 
The axiom of infinity confirms the existence of an infinite set, with the smallest set satisfying these properties being $\{ \emptyset, \{\emptyset\}, \{\emptyset, \{\emptyset\}\},\ldots \}$. 
This is often defined to be the first infinite set of Von Neumann ordinals, which represent the natural numbers. 
The Von Neumann ordinals extend further than this, reaching higher infinities.

\subsubsection*{VII. Axiom Schema of Replacement}
For any set $x$, if $\varphi(a,b)$ is a predicate such that for each $a\in x$, there is a unique set $b$ that satisfies $\varphi$, then there exists a set $y$ containing exactly the sets $b$ such that $\varphi(a,b)$.
$$\forall x: \big{[}(\forall a: a\in x \Rightarrow \exists! b: \varphi(a,b))
  \Rightarrow [\exists y: \forall b: (b \in y \Leftrightarrow \exists a: (a\in x \wedge \varphi(a,b)))]\big{]}$$
In other words, if $\varphi(a,b)$ is a predicate that behaves like a function, and if $x$ is a set acting as the domain of the function, then the sets $b$ that are uniquely determined by the predicate belong to a set $y$.
The axiom schema of replacement is only needed for proving the existence of some very esoteric sets which couldn't be proven to exist otherwise. 

\subsubsection*{VIII. Axiom of Foundation}
If $x$ is a non-empty set, then $x$ has a member $a$ such that $a$ and $x$ have no common member.
$$\forall x: x \neq \emptyset \Rightarrow \exists a: a\in x \wedge a \cap x = \emptyset$$
Axioms I-VII are compatible with the existence and non-existence of self-containing sets, but this axiom forces their non-existence. The axiom of foundation tells us that every set is \emph{well-founded}, that is, there are no infinitely descending chains of sets, such as $\{\{\ldots\{\ldots\}\ldots\}\}$.

Using these axioms, we can go on to define various useful mathematical objects.

\subsubsection*{Ordered Pairs}
\label{zfordpair}
The ordered pair $(a,b)$ is commonly defined as the set $\{\{a\},\{a,b\}\}$, whose existence can be proved using three applications of the axiom of pairing:
\begin{align*}
  \{a\} & \text{  (pairing of $a,a$)} \\
  \{a,b\} & \text{  (pairing of $a,b$)} \\
  \{\{a\}, \{a,b\}\} & \text{  (pairing of $\{a\}$ and $\{a,b\}$)}
\end{align*}
It can then be proved that this definition satisfies the characteristic property of ordered pairs, that is, $(a,b) = (c,d) \Rightarrow (a=c \wedge b=d)$.

Ordered tuples of arbitrary length can be defined by nesting pairs inside of each other:
$$(a,b,c) \equiv_{\mathit{def}} (a,(b,c)) = \{\{a\},\{a, \{\{b\},\{b,c\}\}\}\}$$

\subsubsection*{Union and Intersection}
The union of two sets $x$, and $y$, denoted $x\cup y$ is defined as the set containing all members of $x$, and all members of $y$. Its existence can be proved by the axiom of pairing to create $\{x,y\}$, and the axiom of union to create the desired set $\bigcup \{x,y\}$.

Set intersection, $x \cap y$, is defined as the set all members which are contained in both $x$ and $y$.
The existence of this set can be proved by an instance of the axiom schema of subsets. Letting $\varphi(a) \equiv a \in y$, the specification $x_\varphi$ is then the desired set $x \cap y$.

\subsubsection*{Cartesian Product}
The cartesian product of two sets $x$ and $y$ is defined as the set of all pairs $(a,b)$, such that $a\in x$, and $b\in y$.
By the axiom of pairing, all of the pairs $(a,b)$ exist.
The sets $\{a\}$, and $\{a,b\}$ belong to $\mathcal{P}(x\cup y)$, so any pair $\{\{a\},\{a,b\}\}$ belongs to $\mathcal{P}(\mathcal{P}(x\cup y))$, as well as many other sets which are not ordered pairs.
Using the axiom of specification, with the predicate
$\varphi(p) \equiv \exists a,b: p = (a,b) \wedge a\in x \wedge b\in y$, we can create $\mathcal{P}(\mathcal{P}(x\cup y))_{\varphi}$ to get the desired set, denoted $x\times y$.

\subsubsection*{Relations and Functions}
Generally, relations are given as predicates.
However, only some of these relations can be represented as sets containing only ordered pairs.
We prove that a binary relation $\varphi(a,b)$ can be represented by a set, by proving the existence of a set $z$ containing all $x$ and $y$ such that $\varphi(x,y)$.
Then we can use specification on $z\times z$, to get the desired set $r = \{(x,y) \in z\times z \mid \varphi(x,y)\}$.
This same method of proof generalises to ternary relations, 4-place relations and so on.

Functions can also be modelled as sets.
A set $f$ is a function iff $f$ is a binary relation, and for each $x$, there is at most one $y$ such that $(x,y)\in f$.
A function maps members of a set $a$ --- called the domain --- to the members of a set $b$ --- known as the range.
The existence of these sets can be proved using the axiom of replacement.
Given that the domain $a$ of the desired function exists, and we have a predicate $\varphi(x,y)$ that behaves like a function, we can confirm the existence of the set $b$, and hence $a\times b$.
By specification, $(a\times b)_\varphi$ is the desired set $f$.

\section{Ordered Pairs}\label{OP}
Ordered pairs ---  as frequently mentioned throughout this document --- are objects which hold two elements: a left projection, and a right projection. 
They are used to define co-ordinate systems, relations and functions, and are a key, commonly used mathematical structure. 
Kanamori~\cite{kanamori} gives a fantastic review of the history of the ordered pair. 
He firstly notes that ordered pairs were not explicitly used in the work of Descartes, even though most students' first introduction to ordered pairs will have been in simple high-school geometry.
The first objectification of ordered pairs is attributed to Hamilton in 1837~\cite{hamilton}, using them to represent complex numbers (using $(a,b)$ to represent the complex number $a+bi$).
Peano \cite{peano} was focused on creating economical ``reductions" in mathematics, such as defining \emph{sequences} as functions on the natural numbers. 
He considered ordered pairs to be primitive, and defined relations as collections of ordered pairs. 
He explicitly gave the ``characteristic property"\footnote{This is equivalent to the property given in~\ref{zfordpair} since $(x,y)=(a,b)$ follows from $(x=a\wedge y=b)$ for any definition of the ordered pair}  of the ordered pair:
$$\forall x,y,a,b: (x,y) = (a,b) \iff (x=a \wedge y=b)$$
and noted the possibility of an encoding of the ordered pair as a set.   
However Russell and Whitehead did quite the opposite in their \emph{Principia Mathematica} ~\cite{pm}, deriving the ordered pair in terms of relations. 

Wiener was the first to define the ordered pair as a set, defining $(x,y)$ as
$$\{\{\{x\},\emptyset\},\{\{y\}\}\}~\cite{wiener}$$ 
and notes that this definition satisfies the characteristic property given by Peano.
Around the same time, Hausdorff gave his definition as 
$$\{\{x,1\},\{y,2\}\}~\cite{hausdorff}$$ 
where the objects $1$ and $2$ were required to be distinct objects.
This definition works even when $x$ or $y$ is equal to $1$ or $2$. 

Kuratowski then introduced the now-standard definition given in most textbooks:
$$\{\{x\},\{x,y\}\}~\cite{kuratowski}$$
This satisfies the characteristic property, and is also more economical, in the sense that is does not require any external objects, and a pair's existence can be proven solely from the axiom of pairing.
This definition seemed to stick, as it was adopted by Von Neumann~\cite{von1961}, G\"odel~\cite{godel1992}, Bernays~\cite{bernays1937}, and Tarski~\cite{tarski1931}.  
The projection relations for the Kuratowski defined pairs, $\pleft$ and $\pright$, are defined as:
\begin{align*}
a\pleft p &\eqdef (\forall x\in p: a\in x) \\
b\pright p &\eqdef (\exists x\in p: b\in x) \wedge
                (\forall x,y \in p: x\neq y \Rightarrow (b\notin x \vee b\notin y))
\end{align*}
These definitions, when applied to Kuratowski ordered pairs, can easily be shown to have the properties of uniqueness, so that if $a\pleft p$, and $b\pleft p$, then $a=b$, and similarly for $\pright$. As a consequence, they also have the characteristic property of ordered pairs, so that if $\forall a: (a\pleft p)\Leftrightarrow(a\pleft q)$, and $\forall b: (b\pright p)\Leftrightarrow(b\pright q)$, then $p=q$.\\

Quine, in his published PhD dissertation~\cite{quine1934}, took the ordered pair to be primitive, but still noted the ``Wiener-Kuratowski'' ordered pair. 
Far later, in Quine's philosophical work \emph{Word and Object}~\cite{quineword} in 1960, he describes the Kuratowski ordered pair, but emphasizes that as long as the definition of the ordered pair satisfies the characteristic property, the choice of definition really doesn't matter. 
Quine writes: 
\begin{quotation}
``A similar view can be taken of every case of explication: \emph{explication
is elimination}. We have, to begin with, an expression or form of
expression that is somehow troublesome. It behaves partly like a
term but not enough so, or it is vague in ways that bother us, or it
puts kinks in a theory or encourages one or another confusion. But
also it serves certain purposes that are not to be abandoned. Then
we find a way of accomplishing those same purposes through other
channels, using other and less troublesome forms of expression.
The old perplexities are resolved.''
\end{quotation}
Quine also gave his own definition of the ordered pair in his \emph{New Foundations}, though it is extremely convoluted, and not nearly as economical as the earlier definitions by Wiener, Hausdorff and Kuratowski.
It does however, have the advantage of being \emph{type-level}. 
Russell's theory of types was intended to avoid paradoxes, so that in statements such as $x\in y$, $x$ must be one type lower than $y$. 
A modern interpretation of Russell's type is the \emph{rank} of a set, introduced below in \ref{vonneu}.
It was seen as a problem that the previous definitions of the ordered pair $(x,y)$ were actually two types higher than the type of $x$ and $y$.
This new definition ensured that $(x,y)$ is the same type as $x$ and $y$.

Dana Scott also gives a similar type-level definition in his article \emph{Reconsidering Ordered Pairs}~\cite{dscott}.
This ordered pair is then used to define ordered $n$-tuples.

\section{Model Theory}
First-order model theory is described as a branch of mathematics that deals with the relationships between descriptions in first-order languages, and the structures which satisfy these descriptions~\cite{stanmodel}.
Meaning is given to formula in a first-order language by assigning functions and relations to the symbols in the signature of the language, and by specifying a set that acts as the domain for the variables of the language to range over.

\subsection{Structures}\label{def:structures}
A \emph{structure} \cite[ch.~2.1]{selinger} $\mathcal{A}$ is a triple $(A, \sigma,\mathcal{I})$ where:
\begin{itemize}
  \item $A$ is a set, often called the domain of $\mathcal{A}$, denoted $|\mathcal{A}|$. These are the objects that the variables of formula will range over. 
  \item $\sigma$ is a signature.
  \item $\mathcal I$ is the interpretation function that assigns actual functions and relations to the function and relation symbols in $\sigma$. Each constant symbol is  assigned an element in $|\mathcal{A}|$.
\end{itemize}
Using this definition, we can now formally define the standard group of integers as a structure with the $\sigma_{\mathit{grp}}$ signature:
$$\mathcal{Z} = (\mathbb{Z}, \sigma_{\text{grp}}, \mathcal{I})$$
$$\text{ where \hspace{1mm}} \mathcal{I}(*) = +,\hspace{1mm} \mathcal{I}(^{-1}) = -, \text{ and } \mathcal{I}(e) = 0.$$

\subsection{Satisfiability of Formulas}
We can determine which sentences of the language $\mathcal{L_\sigma}$ are true under the interpretation and domain of a structure $\mathcal{M}$ by inductively evaluating the truth values of formula using Tarski's Truth Schema\cite{tarski}.
A \emph{variable assignment} on a structure $\mathcal{A}$ is a function $\mu: \textbf{Var} \rightarrow |\mathcal{A}|$ that assigns each variable an element in the domain of the structure.
We say that a formula $\psi$ is \emph{satisfiable} if there is a structure $\mathcal A$, and variable assignment $\mu$ in which $\psi$ is true, written as $\mathcal{A}, \mu \vDash \psi$ \cite[ch.~2.3]{selinger}.\\

\noindent
If $\psi$ has no free variables, then the variable assignment does not affect the truth evaluation, and we simply write $\mathcal{M} \vDash \psi$.
In most sentences ``for all" and ``there exists" quantifiers are involved.
So the sentence $\forall x: \psi(x)$ is true if and only if $\psi(x)$ is true for all variable assignments of the variable $x$, and the sentence $\exists x: \psi(x)$ is true if and only if there exists a variable assignment of $x$ such that $\psi(x)$ is true.

\subsection{First-Order Theories}
A \emph{first-order theory} is a set of sentences in a first-order language of a given signature.
A structure $\mathcal M$ satisfies a theory $T$ when all of the sentences of $T$ are satisfied by $\mathcal{M}$, written $\mathcal{M} \vDash T$. If there is a structure that satisfies a theory, then the theory is called \emph{satisfiable}, moreover, the structure is called a \emph{model} for $T$ \cite[ch.~2.5]{selinger}.

Any sentences that follow from the sentences of the theory via logical consequence are called the \emph{theorems} of the theory. We write $T\vdash \psi$ if a sentence $\psi$ is a theorem of the theory $T$.

Theories can be specified by constructing a structure $\mathcal{M}$, and considering the theory to be all of the sentences satisfied by $\mathcal{M}$. This is called the \emph{complete theory} of $\mathcal{M}$, denoted by $\mathit{Th}(\mathcal{M})$ \cite[ch.~1]{stanmodel}.
Alternatively, we could first list an initial set of axioms, then let the theory $T$ be the smallest set containing the axioms, that is also closed under logical consequence. The class of all structures which are models of this theory, denoted $\mathit{Mod(T)}$, is said to be \emph{axiomatised} by $T$.
When a theory $T$ is constructed with the purpose of describing a structure $\mathcal{M}$, we say that $\mathcal{M}$ is the intended model of $T$ \cite[ch.~2.2]{shortermodel}.

\subsection{Models of $\mathit{ZF}$ and the Von-Neumann Universe}\label{vonneu}
We talk about the intended model of $\mathit{ZF}$, and also models of other theories and restrictions of $ZF$, by considering the \emph{Von-Neumann universe} $V$ as the domain of the model. This universe is defined using what is known as \emph{transfinite recursion}, a generalisation of standard recursion on the natural numbers, that uses cardinal and ordinal numbers instead. The universe is defined in stages, indexed by the class of ordinal numbers:
\begin{align*}
  V_0 &\eqdef \emptyset \\
  V_{\beta+1} &\eqdef \mathcal{P}(V_\beta)\\
  V_{\lambda} &\eqdef \bigcup_{\beta<\lambda} V_\beta
\end{align*}
where $\beta$ is any ordinal number, and $\lambda$ is any limit ordinal.
Each of these stages is a set, since the only operations used are power set and union, which are allowed by the axioms of $\mathit{ZF}$. However, we define the universe itself by taking the union of all stages:
$$V \eqdef \bigcup_\alpha V_\alpha$$
The \emph{rank} of a set is the least ordinal number greater than the rank of any member of the set .

However, $V$ is too large to be a set in $\mathit{ZF}$.
Instead, this is what is known as a \emph{proper class}.
A proper class is a collection of sets, which cannot be constructed as a set through the axioms of $\mathit{ZF}$.
Thus $V$ is the class of all well-founded sets, and is often used as a domain for models of set theory, and is usually assumed to be contained in some more powerful theory that allows the use of classes.

This means that $\mathit{ZF}$ is not powerful enough to model itself, that is, there cannot exist a model for the theory $\mathit{ZF}$ whose domain and interpretation exist as sets in $\mathit{ZF}$.
So when we talk about models of $ZF$, we assume that we are working \emph{outside} of $\mathit{ZF}$, in a more powerful system. A more detailed summary of these issues is given in a paper by Chow~\cite{force}.

\section{Alternative Axiomatic Set Theories}
A comprehensive review of alternative axiomatic set theories is outlined by Holmes, Forster and Libert~\cite{ast}.
As with ZF, most of these theories are designed to ease theoretical work rather than practical use, or use in education.
Two concrete examples of alternative theories being created for some manifesto are detailed below:

Holmes proposes Pocket Set Theory~\cite{pocket} as an alternative foundation for mathematics, with the main difference with respect to other theories being that the only infinite cardinals in his theory are $\aleph_0$ and $c$, which are, respectively, the cardinality of the natural numbers, and the cardinality of the real numbers.
Holmes claims that these are the only two infinities which occur naturally outside of set theory, and that in ZF and other theories, far more superstructure is produced than needed to support classical mathematics.

Another approach to an alternative to ZF as a mathematical foundation is Chiron, given by Farmer \cite{chiron}. Chiron is a derivative of Von-Neumann-G\"odel-Bernays (NGB) set theory, and is intended to be a logic for mechanising mathematics. Farmer notes that ZF, NGB and other such systems are intended to be theoretical tools, which are incredibly expressively theoretically, but have some practical difficulties.

\section{Urelements}
Urelements are objects in a formal set theory that are considered to be distinct from sets.
They contain no elements via the standard membership relation, but can still belong to sets, and they are not equal to the empty set. 
Formal modifications to axiomatic theories need to be made to allow the presence of urelements.
For example, the axiom of extensionality in $\mathit{ZF}$ would say that all urelements are equal to each other, and equal to the empty set, preventing them from existing. 

Zermelo's original axiomatisation did allow the existence of urelements \cite{zermelo}.
Later however, Fraenkel specifically mentioned the unnecessity of them, and this was realised by mainstream set theory.

Barwise \cite{barwise} gives a strong argument for the use of urelements in set theory and why ZF is ``too strong", claiming that large parts of mathematical practice are distorted by the demand that all objects be realised as sets, as opposed to being isomorphic to sets.
He also notes, as above, that in the work of Zermelo, urelements were an integral part of the subject.

Urelements are commonly used as `atoms', that have no observable internal structure.
It seems that defining relations on urelements to give them structure has not been considered.

In the appendix of \cite{barwise}, Barwise also shows us how to formally interpret one set theory with urelements, KPU, in terms of another, KP.
He does this by defining predicates in KP for identifying urelements and sets as follows:
$$U(x) \equiv_{def} \exists y: x = (0,y)$$
$$\text{Set}(x) \equiv_{def} \exists y:(x = (1,y) \wedge (\forall z\in y: U(z) \vee \text{Set}(z))$$
Since for each set in KP $x$, there will exist ordered pairs $(0,x)$ and $(1,x)$, so effectively, two copies of the universe are created, whose objects are distinct from each other.
This allows us to interpret each urelement $u$ in KPU, as the ordered pair $(0,u)$ in KP, similarly each set $x$ as the ordered pair $(1,x')$, where $x'$ is modified so each of its elements are also of the `new' form.
The second part of the conjunction of $\text{Set}(x)$ ensures that all elements of the set are of the same form of the paired universe.
A new definition is then given for set membership in KPU:
$$x\in_{KPU} y \equiv_{def} \exists z: (y=(1,z) \wedge x\in z)$$
From this, we can create a new structure, whose domain ranges over the objects of KP that satisfy $U(x) \vee \text{Set}(x)$, and whose membership relation is $\in_{KPU}$.
Barwise then shows that under this new interpretation, every axiom of KPU is a theorem of KP.
A similar method is given by L\"owe \cite{lowe}, for interpreting ZFU in ZF.
This method will be employed when creating the model for our new theory, but working axiomatically within ZF rather than KP.\\

Dubois~\cite{dubois} presents a set theory which introduces a new constant, $\Lambda$, which represents the concept of ``nothing'', yet is distinct from the empty set.
This object is contained in every set --- $\forall x:x\neq\Lambda \Rightarrow \Lambda \in x$ --- and is a member of no set --- $\forall x: x\in \Lambda$.
This allows the definition of the empty set as the singleton set of $\Lambda$.
The distinction between the empty set and any urelements follows clearly from this introduction\footnote{The original problem stems from the fact that both urelements and the empty set have no members. Now the empty set is actually a set containing $\Lambda$, and urelements still have no members.}.
He gives a new theory by extending any set theory which satisfies extensionality, existence of the empty set, and pairing (this could easily be $\mathit{ZF}$).
He gives new axioms in terms of a new language, which is just the standard language with a new constant $\Lambda$ added. 
No modifications to the original axioms are needed.
A model is created in a similar fashion to Barwise, however, less detail is given.
A new membership relation is defined on this new universe, and it is shown that the new model interprets the extended theory.

\section{Aczel and Lunnon's Generalised Set Theory}
Mid-way throughout the project, we found various pieces of literature by Aczel and Lunnon, that sought to provide a mathematical foundation for the field of Situation Theory.

We were initially interested in this literature upon finding a note on Aczel's website~\cite{aczelsite} reading: 
\begin{quotation}
\begin{small}
\noindent
``This [Generalised Set Theory] is axiomatic set theory, modified to allow for other kinds of objects besides pure sets. The simplest modification is reasonably familiar. This is to allow for atoms (also called urelemente). These are objects that have no internal structure. As well as sets having sets as elements they may have atoms also as elements. But it is also possible to allow for non-sets that do have internal structure. For example axiomatic set theory can be modified so as to allow for a primitive notion of ordered pair. So given any two objects a,b of the universe, sets, atoms or whatever, as well as being allowed to form the unordered pair set {a,b}, it can be allowed to form a new object (a,b), called the ordered pair of a,b. This object is not a set but, like {a,b} is a structured object having internal components a,b. More elaborate kinds of structured objects that are not sets can also be allowed."
\end{small}
\end{quotation}
Tracking down the literature on \emph{Generalised Set Theory} led us to a paper~\cite{gst} detailing this set theory, but not the axiomatisation of it, nor a modification of $ZF$. 
This paper references a series of papers published by Aczel himself, as well as a joint paper between Aczel and Lunnon, as well as Lunnon's PhD thesis~\footnote{We were also unable to obtain a copy of Lunnon's PhD thesis.}.
These papers took an algebraic approach to formulating a generalised theory of structured objects.

The work looks somewhat similar in that they have various different kinds of non-set objects, that each have their own ``components" function. 
However, all of this is defined algebraically, rather than in any set theory. 
There is discussion of axiomatisation of this theory, but little detail is given. 

Discussions with my supervisor found that the literature was extremely hard to read, and we didn't gain much insight into how it could be used for this project.
It is possible however, that Aczel and Lunnon have already solved this problem of primitive ordered pairs, and it is just unclear from the papers presented. 

\chapter{Design and Implementation}
\section{The Theory $\mathit{ZFP}$}
We propose the alternative set theory \emph{Zermelo-Fraenkel Set Theory, with Ordered Pairs} (hereinafter $\mathit{ZFP}$).
This theory is mainly just $\mathit{ZF}$, but has ordered pairs as a primitive structure, as well as sets.
To allow this, we introduce two new relation symbols --- $\pleft$ and $\pright$ --- whose intended meanings are respectively the left and right projection relations.
We formalise this set theory as a list of axioms that characterise the behaviour of the relation symbols. 

\subsection{Requirements}\label{zfpreq}
The main desired property that the projection relation symbols $\pleft$ and $\pright$ should have is:
$$a\pleft p \wedge b\pright p \iff p = (a,b).$$ 
Since we are using relation symbols rather than function symbols, we also require that the projections are \emph{unique}, and that for any ordered pair, its left and right right projections \emph{always exist}.
That way, we ensure that no ordered pairs have more than one left or right projection, and that there is no such thing as an ``empty pair'' or a pair with only one projection. 

The urelements most commonly discussed are objects with no elements via the membership relation, yet still distinct from the empty set.
For this reason, these objects with no internal are sometimes called \emph{atoms}. 
We use the general term \emph{object} to mean something that is either a set, or an ordered pair.
We adopt this vocabulary from this point forward.
Our ordered pairs are technically urelements, in the sense that they are distinct from sets.
In order to make statements about sets and ordered pairs specifically, it is a requirement that our language needs two unary predicates to distinguish between them: 
$$U(p) \iff \text{``$p$ is an ordered pair".}$$
$$Set(x) \iff \text{``$x$ is a set".}$$
Obviously, ordered pairs still have some internal structure, which is endowed via the left and projection relations.
To ensure that our urelements are distinct from sets, we require the property that no ordered pairs have any elements via the set membership relation.
Stated formally, this is: 
$$\forall p: U(p) \Rightarrow \forall a: a\notin p$$

Since these ordered pairs are not sets, their behaviour and existence is not determined by the current axioms of $\mathit{ZF}$.
They will need their own axioms to govern their use.
On top of this, we have the problem that a lot of the axioms of $\mathit{ZF}$ use universal quantification $(\forall)$.
This means that the statement that follows the quantifier must hold for all objects of the theory, which clearly won't be the case for ordered pairs. 
It is thus a requirement to modify the axioms of $\mathit{ZF}$ so that they only hold for the sets of the new theory $\mathit{ZFP}$. 

We then require some basic properties for the ordered pairs in our theory. 
Firstly, we must be able to prove the characteristic property of ordered pairs, that is:
$$(a,b) = (c,d) \Rightarrow (a = c \wedge b = d)$$
In addition to thus, we must be able to prove that all of the ordered pairs that can be proven to exist using the Kuratowski definition in $\mathit{ZF}$, can be proven to exists in $\mathit{ZF}$.

In order to find a model similar to the Von-Neumann universe, we must ensure that all of the objects of our theory are well-founded. 
This means that we must be able to show that there are no infinitely descending chains of sets, or pairs, or a combination of the two. 
 
\subsection{Development}\label{zfpaxioms}
We begin by extending the current language of the theory of $\mathit{ZF}$ which contains a single binary predicate symbol for the set membership.
We add symbols $\pleft$ and $\pright$ for the projection relations on ordered pairs. 
\begin{definition}
The signature of the language of $\mathit{ZFP}$ is defined as:
$$\sigma_\mathit{ZFP} = (\emptyset,\{\in, \pleft, \pright\},\{(\in,2),(\pleft,2),(\pright,2)\})$$
\end{definition}
\noindent
Just as the axiomatisation of the membership relation entirely represents the notion of set, we will use the projection relations in a similar way, to characterise the behaviour and existence of ordered pairs.
We will consider the language of our set theory to be the first order language of signature $\sigma_\mathit{ZFP}$.

Projections of ordered pairs are usually described using function symbols, which take a pair, and return a projection.
Instead, we opt for the use of relation symbols as a design choice. 
If we were to use function symbols instead, then we would be assuming the uniqueness of the projections, therefore assuming a more powerful form of first-order logic.
Relation symbols are also more in the spirit of set theory, since $\mathit{ZF}$ only uses a single relation symbol. 
Another advantage of this is that we are then able mimic proofs of $\mathit{ZF}$, since the projection relations are somewhat analogous to the membership relation.
This facilitates the aim of the new theory being $\mathit{ZF}$-like.  

We do not have a relation symbol for identifying sets and ordered pairs, since we will show that the predicates $U$ and $\mathit{Set}$ can be defined with formula using the projection relations.
We will use the $\mathit{U}$ and $\mathit{Set}$ predicates for restricting quantifications over the domain of discourse to only ordered pairs, or to only sets.
This is similar to when we use the membership relation to only quantify over elements of a certain set.
For example, the formula $\forall x: (x\in A \Rightarrow \varphi)$ is often denoted by $\forall x\in A: \varphi$.
Adopting this convention, we can write formula that only quantify over sets, or ordered pairs.
\begin{definition}
The quantifiers $\all{Set}, \all{U}, \ex{Set}, \ex{U}$ are defined as follows:
\begin{align*}
  (\all{Set} x: \varphi) &\eqdef (\forall x: \mathit{Set(x)} \Rightarrow \varphi)\\
  (\all{U} x: \varphi) &\eqdef (\forall x: U(x) \Rightarrow \varphi)\\
  (\ex{Set} x: \varphi) &\eqdef (\exists x: Set(x) \wedge \varphi)\\
  (\ex{U} x: \varphi) &\eqdef (\exists x: U(x) \wedge \varphi)
\end{align*}
\end{definition}
\noindent
An additional convention is also given by Barwise relating to the use of symbols for variables standing for sets and urelements \cite{barwise}.
The symbols $x,y,z,\ldots$ are used for sets, $p,q,r,\ldots$ for urelements, and $a,b,c,\ldots$ for objects in general.
This convention will be adopted, but it should be noted that the use of these symbols is only for visual aid, and does not imply that the objects themselves are actually sets, or urelements.

As mentioned earlier, we can specify a first-order theory either by creating a model in a theory such as $\mathit{ZF}$ and trying to axiomatise it.
Alternatively, we can create a list of axioms which state the desired behaviour of our relation symbols, and search for models within $\mathit{ZF}$ that satisfy these axioms, so that results in our new theory are consistent with $\mathit{ZF}$.
We attempt to be economical in that we list a minimal number of axioms. 
However finding a minimal core of axioms is left to future work.
It is enough to simply list various properties that are required, and that will be useful for common set-theoretical reasoning. 

\subsubsection*{I. Axiom of Projections}
Firstly, we give an axiom to characterise the ordered pair by existence of a first projection, and a second projection.
\begin{enumerate}[label=(\roman*)]
\item $\forall x: (\exists a: a\pleft) x \Leftrightarrow (\exists b: b\pright x)$
\end{enumerate}
From this axiom, it follows that no object has a left projection, but no right projection, and vice versa.
This makes for a convenient definition for the ordered pair unary relation, which is characterised by the existence of its left and right projections.

\begin{definition} Any object which has a left projection, or equivalently, a right projection, is defined to be ordered pair:
\begin{align*}
  U(x) \eqdef &\exists a: a\pleft x \\
       \eqdef &\exists b: b\pright x
\end{align*}
\end{definition}
\noindent
By this definition, anything that has two projections is an ordered pair. We then define anything that isn't an ordered pair --- that is, anything that has no projections --- as a set.
This leaves the potential for an object to have no projections, and no members --- the empty set.
\begin{definition} Any object which is not an ordered pair, is defined to be a set:
$$\mathit{Set}(x) \eqdef \neg\mathit{U}(x)$$
\end{definition}
\noindent
Next, two additional parts of the axiom are given to ensure the uniqueness of the projections, and the non-existence of any elements via set membership.
\begin{enumerate}[resume, label=(\roman*)]
  \item $\all{U} p: (\exists!a: a\pleft p) \wedge (\exists!b: b\pright p) \wedge (\forall a: a\notin p)$
\end{enumerate}
Now that we have axiomatised the structure of the ordered pair, the distinction of ordered pairs and sets, we can run through the axioms of $\mathit{ZF}$ and make adjustments where necessary.

\subsubsection*{II. Axiom of Extensionality}
The original axiom of extensionality must be changed, since both the empty set, and any ordered pair would be considered equal under the original axiom, since they both contain no members.
Thus we must have an axiom in two parts, one that determines the equality of sets, and one that determines the equality of ordered pairs.
\begin{enumerate}[label=(\roman*)]
\item $\all{Set} x,y:
        (\forall a:
          a\in x \Leftrightarrow a\in y) \Rightarrow x=y$
\item $\all{U} p,q: ((\forall a: a\pleft p \Leftrightarrow a\pleft q)
             \wedge (\forall b: b\pright p \Leftrightarrow b\pright q))
             \Rightarrow p=q$
\end{enumerate}
From these axioms, the identity of both sets and ordered pairs are respectively determined by their members, and their projections. 

\subsubsection*{III. Axiom of Pairing}
The axiom of pairing of $\mathit{ZF}$ allows us to create a set containing two given elements.
This axiom can remain unchanged, it states the existence of an object which contains two elements via the membership relation.
Therefore, the object must be a set.

We add an analogous axiom for creating ordered pairs, similarly with no need to assert that the quantified object is an ordered pair, because of the use of the projection relations.
\begin{enumerate}[label=(\roman*)]
\item $\forall a,b: (\exists x: \forall c:
          c\in x \Leftrightarrow (c=a \vee c=b))$
\item $\forall a,b: (\exists p: a\pleft p \wedge b\pright p)$
\end{enumerate}

\subsubsection*{IV. Axiom of Union}
The axiom of union can stay mostly the same, with the small change of only quantifying over sets.
This means that we can only take the union of sets, and not ordered pairs, since we can find the ``union" of an ordered pair using the axiom of projections, pairing, and union.
$$\all{Set}x: [\exists y:\forall a:
    (a\in y \Leftrightarrow (\exists z: z\in x \wedge a\in z))]$$
Note that for sets containing ordered pairs, the union axiom states the existence of a set containing all of the elements of the sets, ignoring the ordered pairs.
This turns out to be a good design choice, as it makes the axiom easier to prove within a model later.
For example, if $x=\{\{a,b\},\{c\},(d,e)\}$, then the union axiom states the existence of $\bigcup x=\{a,b,c\}$.

\subsubsection*{V. Axiom of Power Set}
We must first change how we define the subset relation, since under the original definition, any ordered pair would be considered a subset of any set, and the empty set would also be a subset of any ordered pair.
The definition must be changed so that both objects need to be sets.

\begin{definition} A set $x$ is called a subset of a set $y$ iff $y$ contains all of the elements that are in $x$:
$$x\subseteq y \eqdef \mathit{Set}(x) \wedge
               \mathit{Set}(y) \wedge
                \forall a: (a\in x \Rightarrow a\in y)$$
\end{definition}

\noindent
With this new definition, the axiom of power set remains the same, other than that the power set only exists for sets.
$$\all{Set}x: [\exists y:\forall z:
    (z\in y \Leftrightarrow z\subseteq x)]$$
All of the members of any power set can be proved to be sets, since $z\in \mathcal{P}(x)$ iff $z\subseteq x$, which is only true if $z$ is a set.

\subsubsection*{VI. Axiom Schema of Specification}
The axiom schema must be modified, again with the change that only sets can be specified. 
In the case where no elements of $x$ satisfy the predicate $\varphi$, we want to make sure that this results in $y$ being the empty set, and not anything else, thus we must assert that $y$ is indeed a set.
$$\all{Set}x: [\ex{Set} y:\forall a:
    (a\in y \Leftrightarrow (z\in y \wedge \varphi(a)))]$$

\subsubsection*{VII. Axiom of Infinity}
The axiom of infinity is the only axiom from $\mathit{ZF}$ that we do not need to change. 
We know that the object $z$ is a set since it contains an element, the empty set. 
However, we do need to change the definition of the $\emptyset \in z$. In $\mathit{ZF}$, the empty set was defined as the object containing no elements, characterised as: $x = \emptyset \iff (\forall u: u\notin x)$, making the empty set unique by extensionality.
Now there are many objects with no elements, we must require that $x$ is a set.
$$\exists z: \underbrace
                {(\ex{Set}x:x\in z \wedge\forall u:u\notin x)}
                _{{\emptyset\in z}}
                \wedge\hspace{0.6mm} (\forall x: x\in z \Rightarrow x\cup\{x\}\in z)$$
An appropriate definition for the statement $x\cup \{x\} \in z$ should also be given. However, this is only a minor issue, and will be left to future work.


\subsubsection*{VIII. Axiom Schema of Replacement}
To convert the axiom schema of replacement to our new system, we make the usual changes of restriction on quantifiers:
$$\forall x: [\forall a: a\in x \Rightarrow \exists!b: \varphi(a,b)]
  \Rightarrow \ex{Set} y:[\forall b: b\in y \Leftrightarrow
                                     \exists a: a\in x \wedge\varphi(a,b)]$$
We require that $y$ is a set, due to the case where $y$ is empty as a result of $x$ being empty.
We do not require that this axiom only applies to sets, since $x$ being an ordered pair would only assert the existence of the empty set, since $x$ has no members.

\subsubsection*{IX. Axiom of Foundation}
This axiom allows us to state the requirement that all sets, and also all ordered pairs are \emph{well founded}.
In $ZF$, this banishes self-containing sets, and infinitely descending membership chains.
We must adapt this axiom to include banish self-projecting pairs, and infinitely descending projection chains.
The axiom is stated as follows:
$$\all{Set} x: [\underbrace{(\exists u: u\in x)}_{x \neq \emptyset}
  \Rightarrow \exists a: (a\in x \wedge
               \forall b: (b\in x
               \Rightarrow \neg(b\pleft a \vee b\pright a \vee b\in a)))]
$$
In the case where $x$ contains only sets, this is equivalent to the original axiom, since for any $a\in x$, $b\pleft a$ and $b\pright a$ are false.

We consider a simple consequence of this axiom, namely the non-existence of a self-containing set.
\theorem There are no self-containing sets, or self-projecting pairs.
\proof First suppose that there exists some set $s$ such that $s\in s$.
Let $X = \{s\}$, then by the axiom, since $X$ is non-empty, it must contain some element $a$, such that for all $b\in X$, neither $b\in a$, $b\pleft a$, or $b\pright a$.
Since $X$ only has one member, it must be the case that $a=b=s$, and in particular, $s\notin s$.
This yields a contradiction from the assumption of a self-containing set, proving their non-existence.
A similar argument can be made for self-projecting pairs.\qed \\

\noindent
We must consider infinitely descending chains of sets, and pairs, and chains involving both pairs and sets, and all of their corresponding relations.
More formally, we are talking about chains of the form:
$$\ldots, a_3 \mathrel{\circ_2} a_2 \mathrel{\circ_1} a_1 \mathrel{\circ_0} a_0$$
Where all $a_i$ are objects, and each $\circ_i$ stands for either $\in, \pleft$, or $\pright$. Potential instances of chains of this form are infinitely descending chain of sets:
  $$\{\{\{\ldots\}\}\}:\hspace{4mm}\ldots s_2 \in s_1 \in s_0$$
Infinitely descending chain of pairs, in the first projection:
  $$(((\ldots, c),b),a):\hspace{4mm}\ldots p_2 \pleft p_1 \pleft p_0$$
Some combination of both sets, and ordered pairs, in both projections:
  $$(\{(b, \{\ldots\})\}, a):\hspace{4mm}\ldots s_3 \pright p_2 \in s_1 \pleft p_0$$

\theorem There are no infinitely descending chains of objects.
\proof Suppose there exists some chain of objects such that $a_{i+1} \mathrel{\circ_i} a_i$ for all $i\in\mathbb{N}$. Let $X = \{a_0, a_1, \ldots\}$, then by the axiom, since $X$ is non-empty, there exists some $a_i \in X$ such that for all $a_j\in X$, it holds that $a_j \notin a_i$, and $\neg(a_j\pleft a_i)$, and $\neg (a_j\pright a_i)$.
But since $a_{i+1} \mathrel{\circ_i} a_i$,
where $\mathrel{\circ_i}$ is either $\in$, $\pleft$, or $\pright$, it must be the case that either $a_{i+1} \in a_i$, $a_{i+1}\pleft a_i$, or $a_{i+1}\pright a_i$, giving contradiction.\qed

\subsubsection*{X. Axiom of Cartesian Product}
In $\mathit{ZF}$, we were able to prove the existence of the Cartesian Product $x \times y$ for any sets $x$ and $y$, by using the axioms of union, power set, and specification.
We could do this because ordered pairs were defined as sets, and the power set allows us to access an ``upper level" of sets to which the required ordered pairs belong.
In $\mathit{ZFP}$, our ordered pairs are primitive, and we have no operation analogous to that of power set.
We propose the introduction of an axiom of cartesian product, so that we can easily define functions and relations.
$$\all{Set} x,y: \ex{Set} z:
 \forall u: (u \in z \Leftrightarrow
 \exists a,b: (a \in x \wedge a\pleft u \wedge b\in y \wedge b\pright u))$$

\subsection{Conclusion}
We have presented an axiomatisation of the desired theory $\mathit{ZFP}$, which can be used to reason about ordered pairs as primitive objects.
The provided reasoning is something that is usually merely assumed, and therefore fills a gap in the literature.
Examples of standard mathematical reasoning in the new theory is given in~\ref{zfpexamples}. 
We now reflect on the success criteria for the axiomatisation presented in~\ref{objectives}.

The understandability of the axioms was ensured by a clear explanation and motivation for each axiom, and any modifications needed. 
The syntax provided for the restricted quantifiers intend to give the reader a better idea of what the axiom says.
For instance $(\all{Set} x: \varphi)$ reads in natural language as ``for all sets $x$, $\varphi$ is true'', yet abbreviates the statement $(\forall x: \mathit{Set}(x) \Rightarrow \varphi)$, which would read as ``for all $x$, if $x$ is a set, then $\varphi$ is true''.
Extra readability is also facilitated by the convention of using the Latin symbols $p,q,r$ for urelements, $x,y,z$ for sets, and $a,b,c$ for general objects.  

The modified axioms of $\mathit{ZF}$ provide a framework for reasoning about sets. 
This contributes to the objective of the theory being powerful and $\mathit{ZF}$-like.
We later show that all of the sets of $\mathit{ZF}$ can be modelled isomorphically as pure sets\footnote{Pure sets are sets that contain no urelements in their transitive closure, see~\ref{embedding}.} of $\mathit{ZFP}$.
However, we were unable to show that all of the sets of $\mathit{ZFP}$ --- including those that contain ordered pairs --- satisfy all of the axioms of $\mathit{ZF}$.
Discussions with my supervisor led us to suspect that it would be possible to provide an argument for this by defining a translation from proofs of $\mathit{ZFP}$ to provable statements of $\mathit{ZF}$.   
Due to time constraints, this will be left to future work. 

The axioms give us the power to reason about primitive ordered pairs, by guaranteeing the existence of objects that have no members, but instead, have projections.
The axiom of pairing of $\mathit{ZFP}$ allows us to create pairs, just as the pairing axiom of $\mathit{ZF}$ allowed us to create Kuratowski pairs. 
To my curiosity and initial surprise, we had to add an axiom of cartesian product to allow us to form sets of ordered pairs.
My supervisor expected this would be a necessary development.
Perhaps a more general, or less powerful axiom could have been used to prove the existence of cartesian products. 
An example of this can be found from Barwise~\cite{barwise}, where he uses $\Delta_0$ separation in the absence of a power set axiom.  

We were unsure if there are any uses of ordered pairs which are available in $\mathit{ZF}$ but not in $\mathit{ZFP}$. 
We hope that the axioms of pairing and cartesian product provide all of the functionality required in mathematics. 

When modifying and designing these axioms, the existence of models satisfying the axioms was kept in mind. 
Evidence of this can be seen from the flavour of the axioms --- we try to stay within the realm of ideas of $\mathit{ZF}$ so that a model is easily proved to exist by the axioms of $\mathit{ZF}$. 

In order to avoid contradictions and paradoxes, each of the axioms were given test cases.
These test cases were sketched, and are not detailed in this text. 
Statements involving universal quantification were tested to ensure that they still remained true when the quantified variable was either set or an ordered pair.
Necessary restrictions were put on the quantifiers in the cases of unwanted behaviour. 

We hope that this theory retains some resemblance of Zermelo-Fraenkel set theory, and that it would suit as a theoretical extension to $\mathit{ZF}$ for those who require reasoning about ordered pairs, but don't want to worry about which encoding they use. 
As mentioned above, the use of relation symbols facilitates this, as does the minimal modification of the original axioms. 

\section{A Model of $\mathit{ZFP}$}
Now that we have stated the axioms of our new theory, we construct a model in which all of the axioms are true, to show that our theory is satisfiable.
Specifically, this means we find a structure $\mathcal{W}$ such that $\mathcal{W} \vDash \psi$, for each axiom $\psi$ of $\mathit{ZFP}$.
Moreover, we provide an informal argument that the existence of a model in $\mathit{ZF}$ implies the consistency of the theory $\mathit{ZFP}$. 

\subsection{Requirements}
When constructing this model, we are using $\mathit{ZF}$ as our meta-theory, and so the definitions given for the domain and relations are defined in $\mathit{ZF}$.
This means that unless specified otherwise, the sets, and operations used in this construction, are those of $\mathit{ZF}$.
In particular, when we refer to ordered pairs in $\mathit{ZF}$, we will use the notation $\pair{a}{b}$ to refer to the Kuratowski-defined ordered pair $\{\{a\}, \{a,b\}\}$, to avoid confusion with the primitive pairs of $\mathit{ZFP}$, denoted by $(a,b)$.

A model for our theory consists of a domain of interpretation in which the objects of the theory range over, and a definition for each relation symbol.
More formally, we define a class $W$, and relations $\zin$, $\zpleft$, $\zpright$. 
The main requirement for the construction of this model is that sets and ordered pairs are distinct, so that the predicates $U$ and $\mathit{Set}$ are meaningful.
The membership relation $a\in x$ must only be true if $x$ is a set, and the projection relations $a\pleft p$, $a\pright p$ must only be true if $p$ is a pair.
A common technical trick, is to represent each set as an ordered pair $\pair{0}{x}$, and each ordered pair as $\pair{1}{p}$.
This way we can easily define the new set membership relation to be false in statements where the object in question is an ordered pair, and \emph{vice versa}. 

In order to satisfy the axioms of $\mathit{ZFP}$, the model must be large enough to contain large infinite sets that are in $\mathit{ZF}$. 
Defining a domain via transfinite induction will allow this, in a similar fashion to the Von-Neumann universe.  

If we have a model that satisfies the axioms of $\mathit{ZFP}$, then we are able to `simulate' $\mathit{ZFP}$ in $\mathit{ZF}$.
The \emph{soundness} and \emph{completeness} of first-order logic tells us that there is a correspondence between true statements which can be derived from a set of axioms, and statements holding in any model that satisfies the axioms. 
Symbolically, if $\mathcal{M}$ is a model satisfying $\Gamma$: 
$$\Gamma \vdash \varphi \iff \mathcal{M}\vDash \varphi $$
This means that statements that can be syntactically proved from the axioms, hold in any model $\mathcal{M}$. 
And conversely, anything that holds in a model $\mathcal{M}$ can be syntactically proved from the axioms. 
We will use these properties to show that the theory is consistent, relative to $\mathit{ZF}$.

\subsection{Development}\label{zfpmodel}

\subsubsection{Domain of Interpretation}
First, we construct a domain of discourse in which all of the objects of our theory belong.

An easy way to do this is to `tag' each object using ordered pairs, so that each set $x$ in $\mathit{ZFP}$ is identified with the pair $\pair{0}{x'}$ in $\mathit{ZF}$, and each pair $(a,b)$ is identified with $\pair{1}{\pair{a'}{b'}}$. To construct a universe where all objects are of this form, we follow a similar construction to the Von-Neumann Universe:

\begin{definition} For each ordinal $\alpha$, the set $W_\alpha$ is defined via transfinite recursion:
\begin{align*}
 W_0 &\eqdef \emptyset\\
 W_{\beta+1} &\eqdef \{0\}\times\mathcal{P}(W_\beta) \cup \{1\}\times W_\beta^2\\
 W_\lambda &\eqdef \bigcup_{\beta < \lambda} W_\beta
\end{align*}
\end{definition}
\noindent
Since the definition is recursive, the existence of each $W_\alpha$ is reliant on the previous. Thus we prove the existence of each $W_\alpha$ via transfinite induction.

\begin{theorem} For each ordinal $\alpha$, the set $W_\alpha$ exists.
  \begin{proof} \hspace{1mm}\\
    \begin{tabular}{p{20mm} p{138mm}}
      $\alpha = 0$: \rule{0pt}{4ex} &
      Using the axiom schema of specification with the predicate $\phi \Leftrightarrow a \neq a$ on any set $x$ gives $x_\phi = \{a \in x \mid a \neq a \} = \emptyset$. \\
      $\alpha = \beta+1$: \rule{0pt}{4ex} &
      Suppose that $W_\beta$ exists. Then $\mathcal{P}(W_\beta)$ and $W_\beta^2$ exist by the axiom of power set, and by the existence of cartesian product. Thus the set $W_{\beta+1} = \{0\}\times\mathcal{P}(W_\beta) \cup \{1\}\times W_\beta^2$ exists by cartesian product, and the axiom of union. \\

      $\alpha = \lambda$ \rule{0pt}{4ex} &
      Suppose that $W_\alpha$ exists for all ordinals $\alpha<\lambda$, then the set $W_\lambda = \bigcup_{\beta < \lambda} W_\beta$ exists by the axiom of union.
    \end{tabular}
  \end{proof}
\end{theorem}
\noindent
The collection of all of these sets cannot itself be a set in $\mathit{ZF}$. However, we can specify the class $W$ by the formula $\Phi_W(x) \eqdef \exists \alpha: x\in W_\alpha$.
We consider this class to be the union of all $W_\alpha$, thus the domain of our model.

\subsubsection{Relation Symbols}
We now go on to define the interpretations of the relation symbols of the signature $\sigma_{\mathit{ZFP}}$.

\begin{definition} The relations $\zin$, $\zpright$ and $\zpleft$ are defined as subclasses of $W^2$:
  \begin{enumerate}[label=(\roman*)]
    \item $\zin = \{(a,x)\in W^2: \exists y: x = \pair{0}{y} \wedge a \in y\}$
    \item $\zpleft = \{(a,p)\in W^2 : \exists q: p = \pair{1}{q} \wedge a\pleft q\}$
    \item $\zpright = \{(a,p)\in W^2 : \exists q: p = \pair{1}{q} \wedge a\pright q\}$
  \end{enumerate}
\end{definition}

\subsubsection{Proving the Axioms}
We now move on to showing that our model $W$ satisfies all of the axioms of our new theory. When proving these axioms, we will be explicit about which relations we are using. We are trying to prove that the relations $\zin$, $\zpleft$, $\zpright$ satisfy the axioms, but they are defined in terms of formula involving the membership relation of $\mathit{ZF}$. We simply run through each of the axioms, with constant discussion:

\begin{enumerate}[series=axiomlist, label=\Roman*.]
  \item \textit{\textbf{Projections:}}
        \begin{enumerate}[series=sublist, label=(\roman*)]
        \item \textit{\textbf{Existence:}} $\forall x: (\exists a: a\pleft x) \Leftrightarrow (\exists b: b\pright x)$
        \end{enumerate}
        \begin{proof}
        Let $x\in W_\alpha$ for some ordinal $\alpha$, and suppose that $(\exists a: a\zpleft x)$.
        By definition of $\zpleft$, $(\exists a: \exists p: x = \pair{1}{p} \wedge a \pleft p)$, then since $x = \pair{1}{p}\in \{1\}\times\mathcal{P}(W_{\beta-1})^2$, $p$ must be an ordered pair with projections in $\mathcal{P}(W\beta-1)^2$. 
        Then $p$ must also have a right projection $b \pright p$, and so $b\zpright p$.
        The converse can be proved similarly.
      \end{proof}
\end{enumerate}

We defined the ordered pair identifier $U(x)$ by the existence of a first projection, or equivalently, the existence of a second projection. Using this definition, and the interpretation, we can prove that $U(x)$ is equivalent to $x$ being of the form $\pair{1}{x'}$, and a similar result for the $\mathit{Set}(x)$ relation.
\begin{lemma} Let $a\in W$, then: $U(a) \Leftrightarrow \exists p: a = \pair{1}{p}$, and $Set(a)\Leftrightarrow \exists x: a = \pair{0}{x}$.
  \begin{proof} Let $x\in W$:

    \begin{tabular}{p{7mm} p{133mm}}
      $(\Rightarrow)$\rule{0pt}{5mm} &
      Suppose that $U(x)$, then by definition, there exists $a$ such that $a\zpleft x$, and so $\exists p: x=\pair{1}{p} \wedge a\pleft p$, and so we have $x=\pair{1}{p}$.
      \\
      $(\Leftarrow)$ &\rule{0pt}{5mm}
      Suppose that $x = \pair{1}{p}$ for some $p$, then $x\in\{1\}\times W_\alpha^2$ for some ordinal $\alpha$, so $p\in W_\alpha^2$.
      Thus $p$ is in fact an ordered pair, and must have a first and second projection, so it holds that $U(x)$.
    \end{tabular}\\

    \noindent
    We now prove the property for sets:
    \begin{align*}
      \mathit{Set}(x) &\iff \neg U(x) \\
                      &\iff \neg (\exists p: x=\pair{1}{p})
    \end{align*}
    But since $x\in W$, either $x = \pair{0}{y}$, or $x = \pair{1}{p}$. So the non-existence of such a $p$ is equivalent to the existence of $y$ such that $x=\pair{0}{y}$.
  \end{proof}
\end{lemma}

\begin{enumerate}[series=axiomlist, label=\Roman*.]
\item \hspace{1mm}
  \begin{enumerate}[resume=sublist, label=(\roman*)]
  \item \textit{\textbf{Uniqueness:}} $\all{U} p: (\exists! a: a\pleft p) \wedge (\exists! b: b\pright p)$
  \end{enumerate}
    \begin{proof}
      Let $p$ be such that $U(p)$, then there exists $q,a,b$, such that $(p = \pair{1}{q}) \wedge (a\pleft q) \wedge (b\pright q)$.
      But since the projections of Kuratowski pairs are unique, it follows that:
      $\exists! q: \exists! a, b:(p=\pair{1}{q})\wedge (a\pleft q) \wedge (b\pright q)$, and so $\exists! a: a\zpleft x$, and $\exists! b: b\zpright x$.
    \end{proof}
  \begin{enumerate}[resume=sublist, label=(\roman*)]
    \item \textit{\textbf{Emptiness:}} $\all{U} p:(\forall a: a\notin p)$
  \end{enumerate}
  \begin{proof}
    Let $p\in W$ such that $U(p)$, then $p = \pair{1}{q}$ for some $q$. Now suppose that $a\zin p$ for some $a$.
    Then it follows that $\exists x: p = \pair{0}{x} \wedge a\in x$.
    Contradiction since $p = \pair{1}{q}$. Thus $\forall a: \neg(a\zin p)$.
  \end{proof}
\end{enumerate}
Now that we have proved these properties relating to the form, existence and uniqueness of projections of ordered pairs, we move on to proving the axiom of extensionality:
\begin{enumerate}[resume=axiomlist, label=\Roman*.]
  \item \textit{\textbf{Extensionality:}}
  \begin{enumerate}[series=sublist, label=(\roman*)]
    \item \textit{Sets:} $\all{Set} x: (\forall a: (a\in x \Leftrightarrow a\in y)) \Rightarrow x=y$
  \end{enumerate}
  \begin{proof}
    Suppose that $x\in W$, and that $\forall a\in W: (a \zin x \Leftrightarrow a\zin y)$
    By definition we have that there exist $x'$ and $y'$ such that $x=\pair{0}{x'}$, $y = \pair{0}{y'}$, and $\forall a\in W: (a\in x' \Leftrightarrow a\in y')$. 
    Since $x'\in\mathcal{P}(W_\alpha)$, and $x'\in\mathcal{P}(W_\beta)$ for some $\alpha, \beta$, it follows that $x'$ and $y'$ only have members in $W$. 
    It then follows that $x'=y'$.
    $$x'=y'\implies \pair{0}{x'} = \pair{0}{y'} \implies x = y$$
  \end{proof}
  \begin{enumerate}[resume=sublist, label=(\roman*)]
    \item \textit{\textbf{Ordered Pairs:}}
    $\all{U}p,q: ((\forall a: a\pi_1 p \Leftrightarrow a\pi_1 q))\wedge(\forall b: b\pi_2 p \Leftrightarrow b\pi_2 q)\Rightarrow p=q$
  \end{enumerate}
  \begin{proof}
  Let $p,q$ be ordered pairs, so $p = \pair{1}{p'}$, and $q= \pair{1}{q'}$, where $p'\in W_\alpha^2$ and $q'\in W_\beta^2$ for some ordinals $\alpha$, $\beta$.
  Suppose that $(\forall a\in W: a\zpleft p \Leftrightarrow a\zpleft q)$, and also $(\forall b\in W: b\zpright p \Leftrightarrow b\zpright q)$.
  By definition of $\zpleft$ and $\zpright$, we have that $(\forall a\in W: a\pleft p' \Leftrightarrow a\pleft q')$, and also $(\forall b\in W: b\pright p' \Leftrightarrow b\pright q')$.
  Since the pairs $p'$, $q'$ only have projections in $W$, the above statement is equivalent to: $$(\forall a: a\pleft p' \Leftrightarrow a\pleft q')\wedge(\forall b: b\pright p' \Leftrightarrow b\pright q')$$
  and so $p'=q'$.
  $$p'=q' \implies \pair{1}{p'} = \pair{1}{q'} \implies p = q$$
  \end{proof}
  \item \textit{\textbf{Pairing:}}
  \begin{enumerate}[series=sublist, label=(\roman*)]
    \item \textit{\textbf{Sets:}}
    $\forall a,b: (\exists x: \forall c: c\in x \Leftrightarrow (c=a \vee c=b))$
  \end{enumerate}
  \begin{proof}
    Let $a,b\in W$ , then $a,b\in W_\alpha$ for some ordinal $\alpha$.
    Let $x = \pair{0}{x'}$, where $x'=\{a,b\} \in \mathcal{P}(W_\alpha)$, and so $x\in W_{\alpha+1}$.
    For all $c$, $c\in x' \Leftrightarrow (c=a \vee c=b)$.
    Then:
      $$c\zin x \iff c\in x' \iff (c = a \vee c = b)$$
  \end{proof}

  \begin{enumerate}[resume=sublist,label=(\roman*)]
    \item \textit{\textbf{Ordered Pairs:}}
    $\forall a,b: (\exists p: a\pleft p \wedge b\pright p)$
  \end{enumerate}
  \begin{proof}
  Let $a,b\in W$ , then $a,b\in W_\alpha$ for some ordinal $\alpha$.
  Now let $p = \pair{1}{p'}$ where $p'=\pair{a}{b} \in W_\alpha^2$, and so $p\in W_{\alpha+1}$, and $a\zpleft p$, $b\zpright p$.
  \end{proof}

\item \textit{\textbf{Union:}}
      $\all{Set} x: \exists y: \forall a: (a\in y \Leftrightarrow (\exists z: z\in x \wedge a\in z))$
  \begin{proof}
%   Let $x=\pair{0}{x'}\in W$ be a set, and let $y=\pair{0}{y'}$, where $$y'=\cup(\cup\cup \{p\in x' : 0\pleft p\} \setminus \{0\})$$
%   Then:
%   \begin{align*}
%     a\zin y &\iff (a\in \cup(\cup\cup \{p\in x' : 0\pleft p\} \setminus \{0\})) \\
%     &\iff (\exists u: u\in \cup\cup \{p\in x' : 0\pleft p\} \setminus \{0\} \wedge a\in u) \\
%     &\iff (\exists u: (\exists v: v \in \cup \{p\in x' : 0\pleft p\} \wedge u \in v) \wedge u \neq 0 \wedge a\in u) \\
%     &\iff (\exists u: (\exists v: (\exists w: w\in\{p\in x' : 0\pleft p\}\wedge v \in w) \wedge u \in v) \wedge u \neq 0 \wedge a\in u)\\
%   \end{align*}
%   So if $w\in\{p\in x' : 0\pleft p\}$, $w=\pair{0}{w'}=\{\{0\},\{0,w'\}\}$ for some $w'$.
%   Then if $v\in w$, then either $v=\{0\}$, or $v=\{0,w'\}$.
%   But $u\in v$, and $u\neq 0$, so $u=w'$.
%   Finally $a\in u$, so $a\in w'$, and $a \zin w$, and $w \zin x$.\\

%   We are also required to show that $y\in W$, which can be seen easily since $x\in W_\alpha$ for some ordinal $\alpha$, then if $a\zin y$, there exists $v\in x'$ such that $a\zin v$.
%   Since $v\in x'$, and $x'\in\mathcal{P}(W_{\alpha-1})$, $v\in W_{\alpha-1}$, thus $a\in W_{\alpha-2}$.
%   So $y'\subseteq W_{\alpha-2}$, and thus $y \in W_{\alpha-1}$.

  Let $x=\pair{0}{x'}\in W_\alpha$.
  If $z\zin x$, and $a\zin z$, then $z\in W_{\alpha-1}$, and $a\zin W_{\alpha-2}$. 
  By the $\mathit{ZF}$ specification schema we prove the existence of the set $y' = \{a\in W_{\alpha-2} : \exists z: z\zin x \wedge a\zin z\}$.
  Letting $y=\pair{0}{y'}$ clearly yields the desired set, and since $y'\in\mathcal{P}(W_{\alpha-2})$, it follows that $y\in W_{\alpha-1}$. 
\end{proof}

\item \textit{\textbf{Power Set:}} $\all{Set}x:(\exists y:\forall z:(z\in y \Leftrightarrow z\subseteq x))$
\begin{proof}
  Let $x\in W_\alpha$ be a set, then $x=\pair{0}{x'}$ for some $x'\subseteq W_{\alpha-1}$.
  Now let $y=\pair{0}{\{0\}\times\mathcal{P}(x')}$, then $y\in W_{\alpha+1}$.
  We must show that $\forall z: (z\zin y \Leftrightarrow z\mathrel{\widehat\subseteq} x)$.
  Recall that the $\mathit{ZFP}$ axiom of power set required a new definition of the subset relation, and also that the relation symbols in this definition fall under the interpretation of the membership relation, so:
    $$z\mathrel{\widehat\subseteq} x \iff \mathit{Set}(z)\wedge\mathit{Set}(x)
                        \wedge \forall a: (a\zin z \Rightarrow a\zin x)$$
\begin{tabular}{p{7mm} p{133mm}}
  $(\Rightarrow)$\rule{0pt}{5mm} &
  Suppose that $z\zin y$, then $z\in\{0\}\times\mathcal{P}(x')$, and so $z=\pair{0}{z'}$, where $z'\subseteq x'$.
  By definition of $\subseteq$, $\forall a:(a\in z' \Rightarrow a\in x')$.
  Since $z=\pair{0}{z'}$, and $x=\pair{0}{x'}$, we have that $\forall a: (a\zin z \Rightarrow a\zin x)$, and also that $\mathit{Set}(z)$, and $\mathit{Set}(z)$. Thus $z\mathrel{\widehat\subseteq}x$.
  \\
  $(\Leftarrow)$ &\rule{0pt}{5mm}
  Now suppose that $z\mathrel{\widehat\subseteq}x$, then $\mathit{Set}(z)$, $\mathit{Set}(x)$, and $\forall a:(a\zin z\Rightarrow a\zin x)$.
  Then $\forall a:(a\in z\Rightarrow a\in x)$, so $z'\subseteq x'$, $z'\in \mathcal{P}(x')$, and $z\in\pair{0}{\mathcal{P}(x')}$.
  Thus $z\zin y$.\qedhere
\end{tabular}\\
\end{proof}

\item \textit{\textbf{Specification Schema:} For any first-order predicate} $\varphi(x)$: \\
$\all{Set}x: (\ex{Set}y: \forall a:
                              (a\in y \Leftrightarrow (a\in x \wedge \varphi(a))))$
\begin{proof}
Since any formula in the signature of $\sigma_{\mathit{ZFP}}$ has an interpretation in the signature of $\sigma_{\mathit{ZF}}$, we can use the $\mathit{ZF}$ axiom schema of specification.
Let $x\in W_\alpha$ be a set, then $x=\pair{0}{x'}$.
Now let $y=\pair{0}{y'}$, where $y'=\{a\in x' : \varphi(a)\}$, and since $y'\subseteq x'$, $y\in W_\alpha$. Then:
\begin{align*}
  a\zin y &\iff a\in y'\\
          &\iff a\in x'\wedge \varphi(a)\\
          &\iff a\zin x' \wedge \varphi(a)
\end{align*}
\end{proof}
\end{enumerate}
Due to time constraints, we did not manage to proof that $\mathcal{W}$ satisfies all of the axioms of $\mathit{ZFP}$. 
The full proofs for the axioms of infinity, replacement, and foundation will be left to future work.
Informal arguments are presented below: 

\begin{enumerate}[resume=axiomlist, label=\Roman*.]
\item  \textit{\textbf{Infinity: }} 
$\exists z: (\emptyset\in z \wedge\hspace{0.6mm} (\forall x: x\in z \Rightarrow x\cup\{x\}\in z))$ \\ 
The design of the domain $W$ is similar to that of $V$, and if $V$ has been shown to satisfy the axiom of infinity, there must be a simple modification of this proof for $W$.

\item \textit{\textbf{Replacement: }}\\
$\forall x: [\forall a: a\in x \Rightarrow \exists!b: \varphi(a,b)]
  \Rightarrow \ex{Set} y:[\forall b: b\in y \Leftrightarrow
                                     \exists a: a\in x \wedge\varphi(a,b)]$\\
The problem with replacement is that it is possible to create the desired set $y$ with the axioms of $\mathit{ZF}$, but I wasn't able to show that it belonged to $W$.
However, I do suspect there would be a simple way of proving this property. 

\item \textit{\textbf{Foundation:} } \\$\all{Set} x: [x\neq\emptyset
  \Rightarrow \exists a: (a\in x \wedge
               \forall b: (b\in x
               \Rightarrow \neg(b\pleft a \vee b\pright a \vee b\in a)))]$\\
The membership and projection relations, $\zin$, $\zpright$ and $\zpleft$ are defined by $\in$, which is well-founded in $\mathit{ZF}$.
Therefore it is sufficient to assume that foundation holds in $\mathcal{W}$.
\end{enumerate}
Finally, we give a proof for the axiom of cartesian product:

\begin{enumerate}[resume=axiomlist, label=\Roman*.]
\item \textit{\textbf{Cartesian Product:} } \\$\all{Set} x,y: \ex{Set} z:
 \forall u: (u \in z \Leftrightarrow
 \exists a,b: (a \in x \wedge a\pleft u \wedge b\in y \wedge b\pright u))$
\begin{proof}
Let $x,y \in W$, then both $x,y\in W_\alpha$ for some ordinal $\alpha$.
Then for any $a\zin x$ and $b\zin y$, we have $a, b \zin W_{\alpha-1}$.
Since $\{1\} \times W_{\alpha-1}^2 \subseteq W_\alpha$, any ordered pair $(a,b)$ is in $W_{\alpha}$. 
Defining the cartesian product via the $\mathit{ZF}$ specification schema gives: 
$$x \widehat{\times}y = \pair{0}{\{p\in W_\alpha : \exists a, b: a\zin x \wedge b\zin y \wedge a\zpleft p \wedge b\zpright p\}}$$
Then, since this set is a subset of $W_\alpha$, it follows that $x\widehat{\times} y \in W_{\alpha + 1}$. 
\end{proof}
\end{enumerate}
Since we did not manage to prove all of the axioms, we will refer to the theory $\mathit{ZFP}$ without the axioms of replacement, infinity, and foundation as $\mathit{ZFP}^{-}$. 
We have now proved that $\mathcal{W} \vDash \mathit{ZFP}^{-}$. 
However, in the following subsection, we will make the safe assumption that $\mathcal{W}$ does indeed satisfy all of $\mathit{ZFP}$. 

\subsubsection{Consistency of $\mathit{ZFP}$}
We now examine a simple consequence of $\mathcal{W}$ satisfying each of the axioms of $\mathit{ZFP}$.
We begin by introducing the co
\begin{theorem}The Soundness Theorem of First-Order Logic\\
Let $\Sigma$ be a set of sentences in first-order logic. 
Let $\mathcal{M}$ be a structure that satisfies $\Sigma$, written $\mathcal{M} \vDash \Sigma$. 
If $\varphi$ is a sentence such that $\Sigma \vdash \varphi$, then $\mathcal{M} \vDash \varphi$.
\end{theorem}

We are able to use this to prove that the consistency of $\mathit{ZF}$ implies the consistency of $\mathit{ZFP}$. 
\begin{theorem}
$$\mathit{Con(ZF)} \Rightarrow \mathit{Con(ZFP)}$$
\begin{proof}
Let $\mathcal{W}$ be the model given above, so $\mathcal{W} \vDash \mathit{ZFP}$.
Then by the soundness theorem, for any sentence $\psi$ such that $\mathit{ZFP}\vdash \psi$, we have that $\mathcal{W} \vDash \psi$. 
Suppose that $\mathit{ZFP}$ is inconsistent, then $ZFP \vdash \varphi,\neg\varphi$ for some formula $\varphi$. 
And so $\mathcal{W} \vDash \varphi, \neg\varphi$.
By definition of the satisfaction relation, we have that $\mathcal{W} \vDash \neg\varphi$ if and only if $\mathcal{W} \nvDash \varphi$.
Supposing the consistency of $\mathit{ZF}$, we cannot have this contradiction, and so it follows that $\mathit{ZFP}$ is also consistent. 
\end{proof}
\end{theorem}

This result is also supported by Enderton~\cite{enderton}:
\begin{quote}
``\textbf{Corollary 25E:} If $\Gamma$ is satisfiable, then $\Gamma$ is consistent."
\end{quote}
\subsection{Conclusion}

\section{Embedding $V$ in $W$}\label{embedding}
\subsection{Requirements}

Since $\mathit{ZFP}$ is itself a set theory, we should want that everything that is true about the sets of $\mathit{ZF}$ should also be true about the pure sets of $\mathit{ZFP}$, that is, those not containing primitive pairs in their transitive closure. To show this, we want to find a correspondence between the sets of $\mathit{ZF}$ and the pure sets of $\mathit{ZFP}$ that preserves truth. This notion of correspondence is captured in a certain kind of map known as an \emph{embedding}.

\begin{definition}\label{embedding}
Let $\mathcal{M}$ and $\mathcal{N}$ be structures of the same signature. An embedding between is an injective map $\alpha:M\to N$ such that for each function symbol, and each relation symbol in the signature:
 $$\alpha(f_{\mathcal{M}}(m_1,\ldots,m_n)) = f_{\mathcal{N}}(\alpha(m_1),\ldots,\alpha(m_n))$$
 $$R_\mathcal{M}(m_1,\ldots,m_n) \iff R_\mathcal{N}(\alpha(m_1),\ldots,\alpha(m_n))$$
\end{definition}

\subsection{Discussion}


\begin{definition}
Define the pure sets of $W$ via transfinite recursion:
\begin{align*}
W'_0 &\eqdef \emptyset\\
W'_{\beta+1} &\eqdef \{0\}\times\mathcal{P}(W'_\beta) \\
W'_\lambda &\eqdef \bigcup_{\beta < \lambda} W'_\beta
\end{align*}
\end{definition}
\noindent
The existence of these sets can be proved via transfinite induction similarly to $W$. Thus we can define the class $W_{\mathit{Pure}}$ by taking the union of all of these sets.
\begin{lemma} For each ordinal $\alpha$, $W'_\alpha \subseteq W_\alpha$.
  \begin{proof} \hspace{1mm}\\
    \begin{tabular}{p{20mm} p{137mm}}
      $\alpha = 0$: \rule{0pt}{4ex} &
      $W'_0 = W_0 = \emptyset$, and clearly $\emptyset \subseteq \emptyset$. \\
      $\alpha = \beta+1$: \rule{0pt}{4ex} &
      Suppose that $W'_\beta \subseteq W_\beta$. Let $x\in W'_{\beta+1}$, then $x = \pair{0}{u}$ where $u \subseteq W'_\beta$. By hypothesis, and by the transitivity of the subset relation, $u\subseteq W_\beta$, so $x\in\{0\}\times\mathcal{P}(W_\beta)$, which is a subset of $W_{\beta+1}$, so $x\in W_{\beta+1}$. Thus $W_{\beta+1} \subseteq W'_{\beta+1}$.
      \\
      $\alpha = \lambda$ \rule{0pt}{4ex} &
      Suppose that $W'_{\beta} \subseteq W_{\beta}$ for all ordinals $\beta < \lambda$. Then, if $x\in W'_\lambda$, then $x\in W'_\delta$ for some $\delta < \lambda$. By hypothesis, $W'_\delta \subseteq W_\delta$, so $x\in W_\delta$, and so $x\in W_\lambda$. Thus $W'_\lambda\subseteq W_\lambda$.
    \end{tabular}
  \end{proof}
\end{lemma}
\noindent
Now that we have confirmed that $W_{\mathit{Pure}}$ is a subclass of $W$, we will define the map from each level of the Von Neumann heirarchy to $W$.

\begin{definition}
For each ordinal $\alpha$, define the map $F_\alpha:V_\alpha \to W'_\alpha$ by:
$$F_\alpha(x) \eqdef \pair{0}{\{F_{\alpha-1}(u)\in W'_{\alpha-1}: u\in x\}}$$
\end{definition}
\noindent
We now prove that this map is well-defined:
\begin{lemma}
For each set $x\in V_\alpha$, there exists a set $y\in W'_\alpha$ such that $F_\alpha(x)=y$.
\begin{proof} \hspace{1mm}\\
  \begin{tabular}{p{20mm} p{137mm}}
    $\alpha = 0$: \rule{0pt}{4ex} &
    For $F_0$, the domain is empty since $V_0 = \emptyset$. So the zero case is trivial.
    \\
    $\alpha = \beta+1$: \rule{0pt}{4ex} &
    Suppose that for each $x\in V_\beta$, there exists $y\in W'_\beta$ such that $F_\beta(x) = y$. Now let $x'\in V_{\beta+1}$, then by definition, $x'\in\mathcal{P}(V_\beta)$, so $x'\subseteq V_\beta$. By hypothesis, for each $u\in x'$, there exists a corresponding $v\in W'_\beta$ such that $F_\beta(u) = v$. We can create the set $y'= \{F_\beta(u)\in W'_\beta : u\in x'\}\in \mathcal{P}(W'_\beta)$,
    and thus the pair $\pair{0}{y'} \in \{0\}\times \mathcal{P}(W'_\beta)$ is, by definition, $F_{\beta+1}(x')$.
    \\
    $\alpha = \lambda$ \rule{0pt}{4ex} &
    Suppose that $F_\beta$ is well-defined for all ordinals $\beta<\lambda$, and let $x\in V_\lambda$. Then since $V_\lambda = \bigcup_{\delta<\lambda}V_\delta$, $x\in V_\delta$ for some $\delta<\lambda$. By hypothesis, there exists $y\in W'_\delta$ such that $F_\delta(x)=y$, and thus $y\in W'_\lambda$.
  \end{tabular}
\end{proof}
\end{lemma}
\noindent
We now prove that $F_\alpha$ is both injective (and therefore invertible), and surjective. Proving both of these properties will show that $F_\alpha$ is a \emph{bijection}.
\begin{lemma} For all ordinals $\alpha$, $F_\alpha$ is injective.
  \begin{proof} \hspace{1mm}\\
    \begin{tabular}{p{20mm} p{137mm}}
      $\alpha = 0$: \rule{0pt}{3ex} &
      For $F_0$, the domain is empty, so the zero case is trivial.
      \\
      $\alpha = \beta+1$: \rule{0pt}{3ex} &
      Suppose that $F_\beta$ is injective, then for all $x,y\in V_\beta$, if $F_\beta(x) = F_\beta(y)$, then $x=y$. Now let $x',y'\in V_{\beta+1}$, and suppose that $F_{\beta+1}(x')=F_{\beta+1}(y')$. Then by definition: $$\pair{0}{\{F_\beta(u) : u\in x'\}} = \pair{0}{\{F_\beta(v) : v\in y'\}}$$
      $$\{F_\beta(u) : u\in x'\} = \{F_\beta(v) : v\in y'\}$$
      By extensionality, these sets must have the same members, so for each $u\in x'$, there is a $v\in y'$ such that $F_\beta(u)=F_\beta(v)$, and then by hypothesis, $u=v$. We have that $u\in x' \Rightarrow u \in y'$, so $x' \subseteq y'$. A similar argument can be made by choosing $v\in y'$, so that $y' \subseteq x'$. Thus $x'=y'$.  \\

      $\alpha = \lambda$ \rule{0pt}{3ex} &
      Suppose that $F_\alpha$ is injective for all ordinals $\alpha < \lambda$. Then let $x,y\in V_\lambda$, and suppose $F_\lambda(x) = F_\lambda(y)$. Since $x,y\in V_\beta$ for some $\beta<\lambda$,  $F_\beta(x) = F_\beta(y)$, and by hypothesis, $F_\beta$ is injective, so $x=y$.
    \end{tabular}
  \end{proof}
\end{lemma}

\begin{lemma} For all ordinals $\alpha$, $F_\alpha$ is surjective.
  \begin{proof} \hspace{1mm}\\
    \begin{tabular}{p{20mm} p{137mm}}
      $\alpha = 0$: \rule{0pt}{3ex} &
      Trivial since $\mathit{Im}(F_0) = \emptyset = W'_0$\\
      $\alpha = \beta+1$: \rule{0pt}{3ex} &
      Suppose that $F_\beta$ is surjective, then for all $y\in W'_\beta$, there exists $x\in V_\beta$ such that $F_\beta(x)=y$. Let $y'\in W'_{beta+1}$ \\

      $\alpha = \lambda$ \rule{0pt}{3ex} &
      Suppose that $F_\alpha$ is surjective for all ordinals $\alpha < \lambda$. Then let $y\in W'_\lambda$, then $y\in W'_\beta$ for some $\beta < \lambda$. Then since $F_\beta$ is surjective, there exists $x\in V_\beta$ such that $F_\beta(x) = y$, and also $x\in V_\lambda$, so $F_\lambda(x) = y$.
    \end{tabular}
  \end{proof}
\end{lemma}
\noindent
Since $F_\alpha$ is both injective and surjective, $F_\alpha$ is a bijection. We now show that $F_\alpha$ preserves the truth of the membership relation, as per definition \ref{embedding}.

\begin{theorem} For all ordinals $\alpha$, for all sets $x,y\in V_{\alpha}$:
  $$x\in y \Leftrightarrow F_\alpha(x) \zin F_\alpha(y)$$
\begin{proof} Let $x,y\in V$. \hspace{1mm} \\

  \begin{tabular}{p{7mm} p{144mm}}
    $(\Rightarrow)$ \rule{0pt}{10ex} &
    Suppose that $x\in y$, then $x\in V_{\alpha-1}$ since $y\subseteq V_{\alpha-1}$.
    By definition, $F_\alpha(y) = \pair{0}{y'}$, where $y' = \{F_{\alpha-1}(u) : u\in y\}$.
    Since $x\in y$, $F_{\alpha-1}(x)\in y'$, but $F_{\alpha-1}(x)=F_\alpha(x)$.
    By definition of $\zin$, $F_{\alpha}(x)\zin F_\alpha(y)$ if and only if $F_\alpha(x) \in y'$, and so $F_\alpha(x) \zin F_\alpha(y)$.
    \\

    $(\Leftarrow)$ &
    Now suppose $F_\alpha(x) \zin F_\alpha(y)$, then $F_\alpha(x) \in y'$, so $F_\alpha(x) = F_{\alpha-1}(u)$ for some $u\in y$, and $F_{\alpha-1}(u) = F_{\alpha}(u)$.
    So by the injectivity of $F_\alpha$, we have that $x=u$, and thus $x\in y$.
  \end{tabular}
\end{proof}
\end{theorem}

\begin{definition} Let $\mathcal{M}$ and $\mathcal{N}$ be structures of signature $\Omega$.
An embedding $\alpha: \mathcal{M} \to \mathcal{N}$ is called elementary if for each formula $\phi(x_1,\ldots,x_n)$ in $\mathcal{L}_\Omega$, and for each $\bar{m} = (m_1,\ldots,m_n)$, we have that:
$$\mathcal N \vDash \phi(\bar m) \Leftrightarrow \mathcal M \vDash \phi(\alpha(\bar m))$$
\end{definition}
\subsection{Conclusion}

\chapter{Evaluation}
\section{Examples of $\mathit{ZFP}$ Set Theory}\label{zfpexamples}
\section{Outcome}
\section{Evaluation of Research Process}

\chapter{Conclusion}
\section{Future Work}
\section{Reflection}


\bibliography{diss}
\bibliographystyle{plain}
\end{document}


% %Dissertation Rubric
% Technical Quality
% • Is the topic meaningful?
%   - Spread out across the document, proof checking etc.
% • Is there evidence of a clear understanding of the project area/research topic?
% • Is there any novelty in the work? Is the work a contribution to the area?
% • Is the topic extensively researched or investigated?
% • Are the project outcomes of a high quality?
%  20 marks

% Project Management
% • Did the student take responsibility for the management of project?
% • Did the student manage their time effectively?
% • Were the project milestones set appropriately and achieved?
% • Was the adopted approach/methodology fully understood and justified?
% • Has the student used appropriate tools/software?
% • Have the appropriate design methodologies been employed?
% • Has the student provided ideas and approaches of original thinking?
%  20 marks

% Results and Evaluation
% • Are the results presented clearly in a logical manner?
% • Are problems and difficulties explained?
% • Does the student demonstrate an understanding and interpretation of results and their significance?
% • Is the application/product complex? Or is it of limited functionality?
% • Does the student demonstrate an appropriate level of understanding of the complexities of the project?
% • Is there any critical evaluation of the project?
% • Has the student suggested future work?
% • Are evaluation and recommendations coherent and logical?
%  20 marks

% Dissertation Document
% • Is the writing clear, concise and with good English?
% • Is the dissertation sensibly structured into chapters and sections?
% • Is the dissertation of an appropriate length?
% • Is the approach adopted well justified?
% • How well did the student discuss and explain their own work?
% • Is the dissertation as a document of a high standard, appropriate for an honours degree?
% • Are the appendices relevant?
% • Does the dissertation exceed the maximum page limit (60 pages excluding front matter and appendices)?
% 20 marks

% Volume of Work and Skill Demonstrated
% • Does the student appear to have undertaken a significant volume of work?
% • Was the student required to master new material to enable them to undertake the project?
% • Is the topic investigated to an appropriate depth?
% • Does the dissertation show a deep understanding of the topic?
